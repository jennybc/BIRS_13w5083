\documentclass[12pt]{amsart}
\oddsidemargin  -0.3in
\evensidemargin -0.3in
\textwidth 7in
\headheight     0in
\textheight=8.5in
\topmargin -0.5in
\headsep 0.5in

\usepackage{color}
%\definecolor{purple}{rgb}{0.65, 0, 0.75}
\usepackage{url}

\newcounter{mynote} \newenvironment{mynote}{\stepcounter{mynote}
\color{red} }{}

\newcommand{\comma}{\marginpar {$\Leftarrow$}}

\title[Statistical Data Integration Challenges in Computational Biology]{PROPOSED
  WORKSHOP ``Statistical Data Integration Challenges in Computational
  Biology: Regulatory Networks Personalized Medicine''}
\begin{document}
\maketitle

\section*{Objectives}
The primary objectives of this workshop are;
\begin{itemize}
\item to identify statistical challenges that arise in joint,
  integrated analyses of biomedical and high-throughput genomic data;
\item to present and critique solutions to these problems, with the
  goal of determining the strategies that will be most effective for
  yielding significant, reproducible biological discoveries;
\item to bring together two communities that are strongly committed to
  tackling the data integration problem: the biologists generating the
  data and the statisticians developing analytical frameworks.
\end{itemize}

In recent years, several large-scale international genomics projects
have been launched.  They include The Cancer Genome Atlas (TCGA)
consortium, the ENcyclopedia Of DNA Elements (ENCODE) and modENCODE
(model organism) projects, and the 1000 Human Genomes project. These
initiatives have now reached a peak in data collection, where massive
amounts of high-quality clinical and genome-scale data have been made
available to the research community. In addition, there exist
several easily accessible WWW databases containing curated biological
knowledge, such as gene functional annotation and gene-protein
interactions, e.g.\ the Gene Ontology (GO) consortium, the Kyoto
Encyclopedia of Genes and Genomes (KEGG), the Human Protein Reference
Database (HPRD), and OncoMine, to name a few.

Joint integrative analysis of these rich data resources remains very
difficult. The obstacles range from the technical (e.g.\ differences
in the nomenclature used to uniquely identify a gene) to the
conceptual (e.g.\ the scarcity of statistical methods that can
reliably and efficiently accommodate highly disparate data types). In
the proposed workshop, we intend to address topics across this wide
spectrum. To ensure that the analytical tools are conceptually sound
and produce results that are biologically interesting and
experimentally verifiable, a collaborative approach is essential.
The workshop is thus designed to foster deeper connections between
``wet-lab'' and ``dry-lab'' researchers and to be a forum for (1) the
dissemination of cutting-edge developments, including new
high-throughput biological assays and novel statistical methodologies,
and (2) the identification of open problems in the analysis of these
diverse data.

\textbf{The main topic for the workshop is data integration and within
  that we target two specific application areas: the identification of
  regulatory networks through multiple data sources and the
  integration of clinical and genome-level data for personalized
  medicine. Finally, we place a special emphasis on studying data
  analysis strategies that promote reproducible research.} We
anticipate that a combination of seminar, poster, and brainstorming
sessions, in which life scientists and quantitative scientists
participate together to flesh out the major issues and roadblocks,
will help clear the way for advances in these targeted areas.

\section*{Relevance, importance, and timeliness}

Relevance: High-throughput biology has changed both genomics and
statistics research. Statistics has become an essential component in
genomics research and analytical challenges from genomics have given
rise to methodological advances.  We are now facing a new challenge:
integration of clinical data with multiple high-throughput and diverse
data sources. Addressing the new statistical demands of data
integration has considerable relevance for continued progress in
biological research and presents exciting opportunities for further
significant methodological advancement.

Importance: With the recent peak in data collection by the
U.S. National Institutes of Health initiatives and international
consortia (e.g.\ TCGA, 1000 Genomes, ENCODE), integration and
statistical analysis of diverse data sources has become a bottleneck
for progress. Yet, it is understood that these data contain
information that can greatly improve our understanding of complex
biological processes, as well as disease causes, prognoses, and
treatments. However, it is no simple task to identify, extract, and
integrate the relevant sets of data for a particular biological
question. The aim of this workshop is for biologists and statisticians
to work together to identify and formulate problems in data
integration with the aim of deliverables that give biological insight
and testable results.

Timeliness: While there are a number of conferences specializing in
analytical challenges that arise in computational biology (e.g. ISMB
and RECOMB), the primary quantitative discipline has been computer
science rather than statistical science.  These meetings are typically
more focused on databases, algorithms, or software, rather than
statistical modeling. Conversely, there are computational biology sessions
at statistical conferences, but the number is small and these conferences are
attended almost exclusively by statisticians.  Opportunities for truly
interdisciplinary interaction between statisticians and biologists are
infrequent but crucially important.

Our aim is to organize an interdisciplinary workshop that addresses
the enormous challenges of statistical data integration in biological
research. Our goal is to focus on the statistical and computational
methods development without losing sight of the underlying biology.
Such a workshop would be an important development in the field and
BIRS provides an ideal environment for such an endeavor.


\section*{Subject area overview}

Recent large-scale initiatives, such as the TCGA consortium, the ENCODE
and modENCODE projects, and the 1000 Human Genomes project, have
produced massive amounts of diverse biological data. Examples of data
sources include: genomic data, such as SNPs (Single Nucleotide
Polymorphism) and CNV (Copy Number Variation); transcription data,
such as mRNA and miRNA expression; and interaction data, such as
information on transcription factor binding sites.  These type of data come
from microarray or, more recently, from next-generation sequencing
technologies (e.g. Illumina, Pacific Biosciences).

The ENCODE project aims at identifying all functional elements in
genome sequences.  Its pilot phase started in 2003 and focused on 1\%
of the human genome. Since then, ENCODE has extended to model organisms
such as mouse (mouse-ENCODE), fruitfly, and worm (modENCODE).  ENCODE
and modENCODE generate genomewide binding profiles of multiple
transcription factors, histone modifications, DNaseI sensitivity, DNA
methylation, copy number variation, and gene expression utilizing
sequencing and array-based technologies.

The Cancer Genome Atlas was launched in 2006. It is now a repository
for thousands of tumor samples from some of the worst prognosis human
cancers, such as glioblastoma and ovarian cancer. The goal of the TCGA
is to use multiple data sources to gain insight into the molecular
bases for human cancers. To this end, TCGA includes both clinical data
(survival, treatment, grade of tumor) as well as genome-level data,
including DNA mutation, copy number variation, alteration in
methylation of the DNA, and mRNA and miRNA expression.

The 1000 Human Genomes project is the most recent of these
initiatives, launched in 2008. In this massive effort, genetic
variation between 2000 individuals from 20 population groups has been
analyzed using next-generation sequencing technologies. In addition,
the project is also a repository for the individual cell-lines, which
opens up for follow-up studies on e.g. mRNA expression and
drug-genome interactions.

All three projects outlined above revolve around the central theme
that to understand a biological process or system we need to take a
global approach and consider all its components.  Below, we briefly
summarize the central problem of data integration in high-throughput
biology. We also describe two targeted areas where data integration
is essential and on which the workshop will focus.

\begin{itemize}

\item \textit{Data integration.}
  
  Disease phenotypes emerge from the joint effects of inherited and
  acquired genetic variation, as well as environmental factors. For
  instance, in cancer tumors it is natural to model the influence of
  point mutations and chromosomal aberrations on mRNA expression, and
  relate these effects to therapy response or patient outcome. We are
  now in a position where we can get data for most components of such
  a model. The goal is to jointly analyze the data sources to identify
  the disease drivers (the genetic variation), and the disease markers
  (expression), as well as the pathways and regulatory processes that
  are altered in the disease.

  Each data source provides a different view of the genome. It seems
  clear, though, that data integration is not very straightforwardly
  applied to data of different types; the most obvious impediment
  being lack of a common parameter across a mix of letter-based
  (sequence), categorical (SNP), ordinal (methylation, protein
  expression), and continuous (expression and copy number) data types.

  One common approach to data integration has therefore been to
  integrate {\em analysis results} (so-called "late data integration")
  rather than the data directly. For example, in analyzing mRNA and
  protein expression data, one can separately identify the mRNAs that
  are differentially expressed between two disease states, and
  similarly for the proteins. A biological pathway can now be ranked
  based on the over-representation of differentially expressed genes
  and proteins within it.

  However, integration of analysis results does not capitalize on the
  informative biological correlation between the different data types,
  results in data reduction (i.e.\ loss of information), and often
  involves manual steps that are not easily reproducible across
  labs. Thus, procedures that simultaneously integrate multiple data
  types into a probabilistic model are needed (so-called "early data
  integration").  We are in the early days of developing such
  procedures, with some promising results appearing in the
  bioinformatics and biostatistics literature (e.g.\ Vaske et al.,
  2010; Shen et al., 2009).

  Both the late and early data integration approaches will be
  well-represented at the workshop. One of the goals of the workshop
  is for life scientists and computational scientists to get an
  opportunity to discuss the relative merits and limitations of the
  above approaches to integration. In addition, we also plan to
  include a session on data integration as pertaining to biological
  knowledge, e.g.\ gene onthology and pathways.  Similarly to the late
  and early integration of experimental data, biological knowledge can
  be used alternatively as post-processing and validation of analysis
  results or as prior knowledge in a biological systems model.  The
  relative merits of these approaches will be debated at the workshop.

\item \textit{Regulatory networks in complex biological systems.}

  A current "hot" topic in systems biology research is the
  identification of regulatory networks and modules.  Network
  reconstruction is also an area with considerable current statistical
  activity. Still, how to integrate several data sources for network
  modeling is a largely undeveloped area. Some efforts have been made
  using so-called late integration, i.e.\ separate estimation of
  network models from each data source and then weighted combination
  of results to produce a final network. Early integration of multiple
  data sources for network reconstruction has just begun to appear in
  the literature, e.g.\ modeling the mRNA concentration of a gene by a
  set of Ordinary Differential Equations, where synthesis and
  degradation rates are allowed to depend on the mRNA concentration of
  other genes and "perturbations" such as copy number variations.

  However, how to expand the scope of both late and early data
  integration to network reconstruction from several data sources
  remains an open problem. If we take cancer as an example, genetic
  variations or perturbations can be thought of as disease drivers. In
  contrast, the impact of these variations on expression, and through
  this on cell-function and disease progression, are comprised by both
  direct and indirect network effects of the disease drivers. By
  including data pertaining to both disease drivers (SNPs, CNV) and
  disease markers (mRNA, miRNA, protein expression), as well as
  clinical data (survival, treatment regime), in regulatory network
  construction, one aims to identify both (1) the main drivers of the
  disease and how they relate to disease progression, and (2) the
  regulatory processes that are most affected and can therefore be
  used as biomarkers or for disease diagnosis.

  The workshop's pre-invited speakers include both experts on
  statistical network reconstruction and life scientists involved in
  biological systems modeling, thus creating an ideal forum for
  discussing the challenges of identification of regulatory networks
  through multiple data sources.

\item \textit{Personalized medicine.}

  Above, we briefly touched on the open problem of how to include
  clinical information in integrative analysis.  The ultimate goal is
  to gain sufficient understanding of how the clinical and molecular
  characteristics of an individual patient's disease are related and,
  consequently, to enable individual patient level-decisions for
  treatment and prognosis, i.e.\ personalized medicine.

  In cancer research and clinical oncology, some degree of
  personalized medicine has been utilized in patient management for
  decades.  However, the variables that have informed these early
  efforts have been crude (e.g.\ tumor's stage, grade, and anatomic
  location and demographic information such as patient's age, risk
  factors, and other basic clinical measures).  In the last decade,
  specific molecular markers have been incorporated. They include PSA
  (prostate-specific antigen) for prostate cancer screening and
  detection and Her2/neu status for determining adjuvant therapy in
  breast cancer patients.

  There is now a flood of new data being collected (discussed
  elsewhere in this proposal) that, if analyzed properly, will lead to
  new knowledge about clinically useful markers and the underpinnings
  of disease. This, in combination with new treatment modalities being
  developed in both the academic and pharmaceutical industry settings,
  has the potential to produce rapid improvements of personalized
  cancer care.

  Two new and exciting trends currently driving personalized medicine
  will be explored at this workshop. The first is extending single
  genomic data type "signatures" to multiple genomic data types and
  integrative analyses. A recent example from The Cancer Genome Atlas
  (TCGA) is the identification of the CpG island methylator phenotype
  in glioblastoma (G-CIMP; Noushmehr et al., 2010). This integrated
  analysis of mRNA expression, methylation and clinical outcome
  identified a subgroup of patients with significantly longer
  survival.  Additional work will be needed to match novel treatment
  strategies with this subgroup.  We are in the early days of such
  findings, so a discussion of methods for discovery will be timely.

  A second trend portends a move away from the paradigm of treatment
  determined by a tumor's tissue-of-origin and toward letting the
  mutational landscape drive treatment decisions. These
  genotype-directed approaches began with targeting of the "gene
  fusion" generated protein BCR-ABL in chronic myelogenous leukemias,
  the first cancer clearly linked to a specific genetic
  variation. Other examples include KIT mutations in gastrointestinal
  stromal tumors, and EGFR mutations in lung adenocarcinomas.
  Recently, (Palanisamy et al., 2010), identified rearrangements of
  the RAF kinase pathway across multiple cancers, thus suggesting an
  unexpected subset of patients for whom RAF inhibitors may be useful,
  even in tumor types that lack prototypical BRAF mutations.

  These examples all highlight the possibilities inherent in a
  multi-faceted approach to data analysis.  We have an ideal group to
  discuss how to identify additional cross-cancer events, as several
  of the participants are experts in cancer genomics. This
  interdisciplinary workshop is also an excellent setting for
  discussing the road blocks in personalized medicine, such as the
  design of clinical studies to validate the findings.
\end{itemize}

The proposed workshop centers on data integration in biomedical and
genomic research. We are also addressing two specific focus areas; (1)
the identification of regulatory modules and networks in complex
biological processes and diseases and (2) personalized medicine. Our
aim is to organize a workshop that addresses the interdisciplinary
needs and challenges of these focus areas. \textbf{Additionally, a
  guiding principle of the workshop is to highlight reproducible
  research, a notion and practice which clearly deserve more attention
  and advocacy in the interdisciplinary field of statistical
  genomics.}
\begin{itemize}

\item \textit{Reproducible research.}

  The complexity of the data analysis that underpins a ``typical''
  genomics paper is extreme.  While this complexity is hardly unique
  to computational biology, the cacophony that arises from the diverse
  disciplines and experimental platforms increases the degree of
  difficulty substantially.  Platform-specific standards have been
  developed for many assays (e.g. the MIAME standard for microarray
  data), but we are just beginning to conceive of similar standards
  for the conduct and reporting of an analysis strategy, especially
  considering the coming wave of publications pertaining to data
  integration.

  There have been several highly-publicized instances where the
  analytical results of genomic studies have been impossible to
  reproduce, even with access to the underlying ``raw data'' and
  software implementations of the analysis methods.  In 2004,
  Tibshirani was unable to replicate the results of Dave et al., who
  claimed to have identified two ``immune response'' gene clusters
  whose expression was predictive of survival in follicular lymphoma
  [see http://www-stat.stanford.edu/~tibs/FL/report/index.html for an
  account of Tibshirani's re-analysis of Dave et al., NEJM Nov 18,
  2004 and the subsequent dialogue in NEJM].  More recently, Baggerly
  and Coombes have highlighted problems in a set of papers from
  investigators at Duke's Institute for Genome Sciences.  This has led
  to the halt of clinical trials, an investigation by the National
  Academy's Institute of Medicine, and patient lawsuits.  [Baggerly
  and Coombes, Ann. Appl. Stat. Volume 3, Number 4 (2009), 1309-1334].

  In both of these cases, the lack of standards for implementing,
  preserving, and disseminating the analysis of genome-scale data was
  a primary factor.  To that end, various members of the community are
  rallying around this issue of ``reproducible research''.  The
  solutions will be multi-level, ranging from software tools and
  practices to top-down prescriptions relating to the release and
  format for raw data, code, and results. General frameworks for
  reproducible research have been proposed in Gentleman and Temple
  Lang (2004). For instance, Sweave (Leisch, 2002), one such system
  applicable to R and \LaTeX, allows the generation of integrated,
  dynamic, and reproducible statistical documents, intermixing text,
  code, and code output (textual and graphical). The document can be
  automatically regenerated whenever the data, code, or documentation
  text change. The reproducible research system (RRS) described in
  Mesirov et al. (2010) is an adaptation of Microsoft Word that links
  to the Broad Institute's GenePattern platform.  While a number of
  statisticians have recently adopted the Sweave system for their
  research projects, the notion and practice of reproducible research
  clearly deserves more attention and advocacy in the
  interdisciplinary field of statistical genomics.
\end{itemize}

This workshop presents an excellent opportunity to construct
guidelines and share solutions to this thorny problem early in the
development of new approaches for the integrated analysis of diverse
genome-scale data types. The planned meeting includes a wide range of
expertise, with both statisticians and biologists exchanging ideas. We
plan to have leading statisticians and biologists give plenary talks
on each focus area. In addition, selected invited speakers will
present their recent efforts on data integration within one of the
focus areas. \textbf{Two informal evening sessions on software and
  databases will be included}, which will be the venue for
highlighting software and practice that promote reproducible
research. Following the successful structure of previous workshops, we
plan to include several poster and brainstorming sessions where
researchers can exchange ideas on how best to respond to the challenge
of data integration and identify open problems and limitations with
the available data and models in use.


\section*{Possible participants and their affiliations}

In addition to the organizers, the following pre-invited participants
have expressed interest and indicated their intention to attend this
workshop. Several of our pre-invited participants are either life
scientists generating one or more of the data sources mentioned in
this proposal or statisticians/computational scientists working on
these data sources.

\begin{itemize}
\item Wolfgang Huber, European Molecular Biology Laboratory, Heidelberg, Germany.
\item Ingo Ruczinski, Department of Biostatistics, Johns Hopkins Bloomberg School of Public Health, Baltimore, USA.
\item Mark Segal, Epidemiology  and Biostatistics, University of California, San Francisco, USA.
\item Dave Stephens, Department of Mathematics and Statistics, McGill University, Montreal, Canada.
\item Raphael Gottardo, Fred Hutchinson Cancer Research Center, Seattle, USA
\item Benjamin Haibe Kains, Dana-Farber Cancer Institute, Harvard
  School of Public Health, Boston, USA.
\item Keith Baggerly, The University of Texas MD Anderson Cancer
  Center, Houston, TX, USA.
\item Jeff Barrett, Sanger
\item Carlos Bustamante, Stanford
\item Goncalo Abecasis, Michigan
\item David Haussler, UC Santa Cruz
\item Peter Campbell, Sanger
\item Nancy Zhang, Stanford
\item Jeff Kidd, Michigan
\item Terry Speed, Wehi
\item Chad Creighton, Baylor
\item Celia Greenwood, Toronto
\item X. Shirley Liu, Harvard
\end{itemize}



\section*{Additional comments}

The organizing committee has been selected to best represent the three
targeted areas of the workshop as well as both communities that are
strongly committed to tackling the data integration problem: the
biologists generating the data and the statisticians developing
analytical frameworks. Several of our members also have experience in
organizing fruitful workshops in the areas of statistics and
statistical genomics.

\begin{itemize}
\item Jenny Bryan is an Associate Professor at the University of
  British Columbia in the Michael Smith Laboratories and Statistics
  Department. She specializes in the analysis of high-throughput
  phenotypic data, such as the datasets being generated through whole
  genome RNAi or in large panels of knockouts. She has co-organized
  five previous successful Statistical Genomics meetings at BIRS in
  2004, 2006, 2007 (2-day), 2008, and 2010. These workshops have
  brought together statisticians, biologists, and clinicians working
  on different aspects of genome-scale studies. They have been hugely
  successful, covering several broad areas of biological investigation
  that relied on statistical and computational methods.

\item Aurelie Labbe is an assistant professor at McGill University in
  the department of Epidemiology, Biostatistics and Occupational
  Health. She works on methodological issue in data integration and
  regulatory networks through the identification of expression
  quantitative trait loci. She has organized the very successful
  international workshop on "Computational statistical methods for
  genomics and system biology" that was held in Montreal in April
  2011. This workshop brought together more than 100 participants from
  Europe and North America, as well as 25 invited speakers
  internationally recognized as leaders in the field.

\item Stephen Montgomery
\item Adam Olshen
\item Ronglai Shen 
\item Paul Spellman 


\end{itemize}


%BIRS (Canada).
%\item Members of the organizing committee have an extensive track
%  record for organizing meetings and conferences on more general
%  statistical inference issues, including invited sessions at the
%  Joint Statistical Meetings, International
%  Biometrics Society ENAR meetings, Institute of the Mathematical
%  Statistics meetings, and New Researchers Conference, as well as
%  several local workshops on statistics and statistical genomics
%  (e.g. Mathematical Sciences Research Institute, Berkeley).
%\end{itemize}

We would like to continue to build on our established track record of
these extremely successful workshops, while adapting to address the
emerging challenge of data integration in systems biology research.

\section*{Dates}

Our preferred week

\subsection*{Preferred dates}
\begin{itemize}
\item June 3-8, 2012.
\item August 13 - 18
\item August 20 - 24
\item July 23 - 27
\end{itemize}

\subsection*{Off-season dates}
\begin{itemize}
\item no strong preferences here?
\item
\end{itemize}


\subsection*{Impossible dates}

\begin{itemize}
\item \textbf{March 10-13, 2013 (ENAR)}
\item \textbf { ???  (ICSA/Applied Statistics Symposium)}
\item \textbf{August 3 - 8, 2013  (IMS/Joint Statistical Meeting)}
\item 
\end{itemize}

\end{document}

%are associated with cancer, such as the regulatory mechanisms and how they are altered by the disease, and functional genetic variations that are the disease drivers.

%Data integration is not an end in itself, however. It is important that the integration and analysis of data lead to testable biological hypotheses and provide biological insight rather than

The integration and joint modeling of diverse data sources is far from trivial. Complications arise from the fact that the data sources have different resolution and dimension (locus, gene, protein, clinical characteristics), and that the different technologies produce varying quality data. LIST TECHNOLOGIES OR NOT?
 Additional obstacles results from the relative incompleteness of the data. That is, some biological samples have been processed with all technologies (genetic variation, gene and protein expression, etc), whereas for other samples the information may restricted to only one or a few data types (e.g.\ only  mRNA expression).  Such complications have as a consequence that no universal method for data integration is easily formulated. To date, data integration is generally applied early or late in the analysis pipeline, with advantages and limitations with either scheme. An early or "low-level" approach to integration involves formulating a biological model that directly relates the data sources to each other. Examples of such approaches are modeling of the change-rate of mRNA expression using differential equations. Such direct modeling of a biological process is advantageous since the results are directly interpretable in a biological context. However, the models are extremely complex when several data sources are considered, and solutions are often intractable unless radical simplifications are made.
A late or "high-level" approach to integration takes the form of weighing or combining evidence from each source. An example is the combination of gene clustering results, where each co-clustering event contributes to a final estimate of a gene-gene interaction. Late integration has the advantage that it can include samples for which only a subset of the data types are available since each data type is analyzed separately. The drawback is, of course, that no direct modeling of the underlying biological process is undertaken, and thus post-processing and interpretation of the results are necessary. In addition, these approaches can be sensitive to the subjective choices of how evidence from different sources are weighted.


%new challenge: deal with diverse data sources, scale-up methods from single microarray experiments to 1000s.
% rapid moving field: need for informal and focused workshops.

Together, these data resources contain a formidable amount of biological information. As an example, TCGA holds thousands of tumors samples from several poor prognosis cancers. In addition to clinical data for the samples, it also provides genome-wide information such as DNA mutation data, Copy Number Variation, mRNA expression, microRNA expression, and the state of methylation. Each data sources provides a different view of the disease, its causes and effects. REPHRASE:
It is only by considering these data jointly that we can begin to understand the complex biological systems and processes whose alterations are a hallmark of cancer.

there are no focused workshops on the important problem of extracting relevant biological information from these massive databases and translate findings to biologically insight or testable and measurable results. REPHRASE: The opportunities
for meaningful quantitative methodological developments are plentiful.

A third example is pharmacogenomics.  The goal is to predict patient response to therapy before it is given.  In order to get the best possible predictions, multiple data types will often need to be utilized rather than just single types such as single nucleotide polymorphisms.



sequence
epigenetic stuff
transcription intensity
protein abundance

does not approach the
  problem in a systems model fashion. A multi-source biological
  systems model can be quite complex to formulate. Successful smaller
  examples exist in the literature, e.g.\ in steady-state modeling of
  change-rate of mRNA expression in response to perturbations such as
  copy number alterations, but often rely on simplifying assumptions
  for tractability purposes. How to expand on such models to include
  more data sources remains an open problem. The relative
  incompleteness of different data sources constitutes an additional
  complication for integrated modeling of the complete biological
  system.


%%%%%%%%%%%%%%%%%%%%%%%%%%%%%%%%%%%%%%%%%%%%%%%%%%%%%%%%%%%%%%%%%%%%%%%%%%%


@article{Noushmehr:2010kx,
	Author = {Noushmehr, Houtan and Weisenberger, Daniel J and Diefes, Kristin and Phillips, Heidi S and Pujara, Kanan and Berman, Benjamin P and Pan, Fei and Pelloski, Christopher E and Sulman, Erik P and Bhat, Krishna P and Verhaak, Roel G W and Hoadley, Katherine A and Hayes, D Neil and Perou, Charles M and Schmidt, Heather K and Ding, Li and Wilson, Richard K and Van Den Berg, David and Shen, Hui and Bengtsson, Henrik and Neuvial, Pierre and Cope, Leslie M and Buckley, Jonathan and Herman, James G and Baylin, Stephen B and Laird, Peter W and Aldape, Kenneth and {Cancer Genome Atlas Research Network}},
	Journal = {Cancer Cell},
	Month = {May},
	Number = {5},
	Pages = {510-22},
	Title = {Identification of a CpG island methylator phenotype that defines a distinct subgroup of glioma},
	Volume = {17},
	Year = {2010}}

@article{Vaske:2010fk,
	Author = {Vaske, Charles J and Benz, Stephen C and Sanborn, J Zachary and Earl, Dent and Szeto, Christopher and Zhu, Jingchun and Haussler, David and Stuart, Joshua M},
	Journal = {Bioinformatics},
	Month = {Jun},
	Number = {12},
	Pages = {i237-45},
	Title = {Inference of patient-specific pathway activities from multi-dimensional cancer genomics data using PARADIGM},
	Volume = {26},
	Year = {2010}}

@article{Shen:2009uq,
	Author = {Shen, Ronglai and Olshen, Adam B and Ladanyi, Marc},
	Journal = {Bioinformatics},
	Month = {Nov},
	Number = {22},
	Pages = {2906-12},
	Title = {Integrative clustering of multiple genomic data types using a joint latent variable model with application to breast and lung cancer subtype analysis},
	Volume = {25},
	Year = {2009}}

@article{Palanisamy:2010vn,
	Author = {Palanisamy, Nallasivam and Ateeq, Bushra and Kalyana-Sundaram, Shanker and Pflueger, Dorothee and Ramnarayanan, Kalpana and Shankar, Sunita and Han, Bo and Cao, Qi and Cao, Xuhong and Suleman, Khalid and Kumar-Sinha, Chandan and Dhanasekaran, Saravana M and Chen, Ying-bei and Esgueva, Raquel and Banerjee, Samprit and LaFargue, Christopher J and Siddiqui, Javed and Demichelis, Francesca and Moeller, Peter and Bismar, Tarek A and Kuefer, Rainer and Fullen, Douglas R and Johnson, Timothy M and Greenson, Joel K and Giordano, Thomas J and Tan, Patrick and Tomlins, Scott A and Varambally, Sooryanarayana and Rubin, Mark A and Maher, Christopher A and Chinnaiyan, Arul M},
	Journal = {Nat Med},
	Month = {Jul},
	Number = {7},
	Pages = {793-8},
	Title = {Rearrangements of the RAF kinase pathway in prostate cancer, gastric cancer and melanoma},
	Volume = {16},
	Year = {2010}}


@TechReport{Gentleman&TempleLang04,
  author = 	 {R. Gentleman and D. {Temple Lang}},
  title = 	 {Statistical Analyses and Reproducible Research},
  institution =  {Bioconductor Project Working Papers},
  year = 	 {2004},
  OPTkey = 	 {},
  OPTtype = 	 {},
  number = 	 {2},
  OPTaddress = 	 {},
  OPTmonth = 	 {},
  OPTnote = 	 {},
  OPTannote = 	 {}
}



@InProceedings{Leisch02,
  author = 	 {F. Leisch},
  title = 	 {Sweave: {D}ynamic generation of statistical reports using literate data analysis},
  OPTcrossref =  {},
  OPTkey = 	 {},
  booktitle = {Compstat 2002 --- Proceedings in Computational
                  Statistics},
  Tpages = 	 {575--580},
  year = 	 {2002},
  editor = 	 {Wolfgang H{\"a}rdle and Bernd R{\"o}nz},
  OPTvolume = 	 {},
  OPTnumber = 	 {},
  OPTseries = 	 {},
  address = 	 {Heidelberg, Germany},
  OPTmonth = 	 {},
  OPTorganization = {},
  publisher = {Physika Verlag},
  note = 	 {ISBN 3-7908-1517-9},
  OPTannote = 	 {},
  url = 	 {www.statistik.lmu.de/\string~leisch/Sweave}
}


@Article{Mesirov10,
  author = 	 {J. P. Mesirov},
  title = 	 {Accessible Reproducible Research},
  journal = 	 {Science},
  year = 	 {2010},
  OPTkey = 	 {},
  volume = 	 {327},
  number = 	 {5964},
  pages = 	 {415--416},
  OPTmonth = 	 {},
  OPTnote = 	 {},
  OPTannote = 	 {}
}

