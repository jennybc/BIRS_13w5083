\documentclass{amsart}

\begin{document}

\title{Proposed workshop ``Statistical Science for Genome Biology''}

\maketitle
\section*{Objectives}

The main objective for this workshop is to facilitate the development
and dissemination of statistical methods relevant to genome-scale
biology, by bringing together leading researchers working at the
interface between the biological and mathematical sciences.  An
important goal, that is reflected in our list of proposed invitees, is
to include statisticians working on different aspects of
\emph{statistics} but all related to \emph{genomics}.  We have
intentionally targeted areas ranging from classical statistical
genetics, such as the genetic mapping of complex traits, to the
emerging area of high throughput gene and protein expression analysis.
It is becoming increasingly important for statisticians working in
genomics to be aware of the quantitative problems and solutions
related to diverse experimental platforms.  Biological investigators
are taking advantage of opportunities to study a system or process
from several angles simultaneously and there is a growing need for
quantitative methods to handle disparate data types seamlessly, for
example, a holistic analysis based on sequence, expression, and
molecular interaction data.  An aim of this workshop is to devise
appropriate statistical formulations for such analyses, which will
expedite the creation and application of sound and powerful
statistical methodologies.

\section*{Relevance, Importance and Timeliness} The utility of
sophisticated statistical methods in modern biology is
well-established and has been addressed in more detail elsewhere in
this proposal.  Promoting genomics as a source of interesting and
important problems is also of great benefit to the statistics research
community.  The fascinating developments in the biological sciences
have generated an unprecedented enthusiasm in the computational
sciences generally (statistics, mathematics, and computer science) by
raising novel and challenging methodological questions.  However,
there can be significant 'barriers to entry and expansion' in this
type of interdisciplinary work. This workshop would enable researching
statisticians, with documented expertise and interest in genomics, to
broaden their knowledge of current research and important open
problems.

The question of timeliness is quite easy to address in this case.  The
pace at which genomic data are accumulating far exceeds the scientific
community's ability to interpret it.  Likewise, the emerging
subspecialty in statistics -- statistical genetics and genomics -- is
growing and changing rapidly and a workshop specifically aimed at this
area is sorely needed.

\end{document}


