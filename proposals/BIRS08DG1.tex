\documentclass[12pt]{amsart}
\oddsidemargin  -0.3in
\evensidemargin -0.3in
\textwidth 7in
\headheight     0in
\textheight=8.5in
\topmargin -0.5in
\headsep 0.5in

\usepackage{color}
%\definecolor{purple}{rgb}{0.65, 0, 0.75}

\newcounter{mynote} \newenvironment{mynote}{\stepcounter{mynote}
\color{red} }{}

\newcommand{\comma}{\marginpar {$\Leftarrow$}}

\title{PROPOSED WORKSHOP ``***EMERGING STATISTICAL CHALLENGES IN GENOME-SCALE DATA ANALYSIS AND TRANSLATIONAL RESEARCH''}
\begin{document}
\maketitle

\section*{Objectives}
The primary objectives of this workshop are

\begin{itemize}
\item to formulate and address emerging statistical problems
in the analysis and combination of diverse types 
of biomedical and high-throughput genome-scale data;

\item to facilitate meaningful interactions
between the experimental biologists and research-oriented clinicans producing genome-scale data 
and the statisticians who will be addressing the issues 
that analysis of these data entails. 
Substantive collaborations between these groups are
vital for transforming the massive amount of data produced by new
technologies into important biological discoveries and translational research leading to the change of paradigm in patient management. 
\end{itemize}

A secondary aim is to honor and celebrate the achievements and ongoing 
contributions to this field of Professor Terry Speed, 
who turns 65 in 2008.
We believe that an appropriate recognition is 
to carry forward his first-rate example of forging productive 
statistical-biological hand-on collaborations,
and that this workshop provides an effective means of doing so.

The workshop is intended to foster deeper connections between the 
biological and statistical research communities and to be a forum for 
(1) the dissemination of cutting-edge developments, including new
high-throughput biological assays and novel statistical methodologies
and (2) the identification of open problems in the analysis of these data.
The challenges include not only analyzing
genotypic, gene expression and protein expression data, 
but also relating these to phenotypic data,
such as biological and clinical outcomes, and further relating
all of these to meta-data from WWW databases such as OncoMine, KEGG, GoldenPath, PubMed, 
the Gene Ontology (GO) Consortium, 
and the Pharmacogenetics and Pharmacogenomics Knowledge Base (PharmGKB). 

We anticipate that this workshop will enable statisticians to
articulate theoretically grounded statistical formulations of existing and
emerging computational biological and clinical problems; create an exceptional
opportunity for exchanging ideas between the communities; and help to
shape the future of this dynamic field. 
Input from biologists and clinicans is absolutely crucial for identification of important questions and development of
appropriate statistical methodologies. 
For this reason, we target areas that are relatively
new to statisticians, as well as areas
that have already been greatly influenced by statistical approaches.

The five targeted areas for the workshop include: patient classification, 
computational population genetics, pharmacogenomics, 
emerging technologies and data integration. 
We expect that the interaction between statisticians
and biologists will lead to major advances in
the analysis and integration of genome-scale data sets and in translational research. 
In addition, this relatively new and rapidly developing field 
enjoys an above-average representation of both young researchers and women.  

\section*{Relevance, importance and timeliness} 

Relevance:  
It is now well accepted that the capacity to generate genome data 
has far outpaced the ability to analyze and interpret it.
The rapid development of new high-throughput technologies
allows investigation of biological processes on an ever-growing scale.
Statistical genomics has adapted well to these changes, 
due to the great interest of statisticians in the methodological challenges
inherent in a quickly evolving domain.
Addressing the new statistical demands has much relevance
for continued progress in biological and biomedical research
predicated on genome-scale assays.

Importance:  
Genome-scale data are rising in prominence, 
and are rapidly becoming critical components of clinical import
in human disease, most strikingly in cancer.
Yet without sound methodology, accompanied by computationally feasible implementations,
we risk missing, or misinterpreting, important information contained in these data.
This workshop will help to enable transformation of the vast data resources 
emanating from multiple, diverse types
of high-throughput assays into realizable health benefits, 
including:  improved diagnostics,
prognostics, risk assessments and treatments.

Timeliness:
There are several well-established computational biology
conferences (e.g., ISMB and RECOMB), where the primary quantitative
discipline has historically been computer science, not statistical science.
These meetings are often focused more narrowly on databases,
algorithmic aspects, or specific software, 
and contain a rather weaker interdisciplinary component.
Likewise, even though major statistical conferences often have
sessions on computational biology, the number of those is still small,
and the audience is almost exclusively statisticians.  
Opportunities for true interdisciplinary interaction are few.
Yet in this field, rapid communication between 
biologists and statisticians is absolutely vital.

Our aim is to organize a more fully interdisciplinary 
workshop to address the challenges posed by the 
enormous need for quantitative data integration and modeling in biology.
Although the field is very broad,
our intent is to focus on the statistical, 
mathematical, and computing aspects without 
losing sight of the underlying biology.
To this end, biologists have an intrinsic role to fulfill.

Achievement of this goal requires input from experts across 
the relevant scientific fields who have a broad vision for the 
global aims in advancing quantitative methods 
for genome-scale data. 
A workshop that specifically brings biologists and statisticians
together would certainly be an important
development in the field, 
and BIRS provides an unbeatable environment for this task. 


%\newpage

\section*{Subject area overview}
Modern high-throughput technologies are changing the face of
biomedical and life science research.
Biological research is moving from a hypothesis-driven
focus on single genes and proteins
to a high-throughput, discovery-driven strategy.
Integrating the vast amounts of ever-changing types of data 
collected to study complicated entities, 
such as protein complexes and regulatory networks,
requires an interdisciplinary approach. 
Cooperation among statistics, mathematics, computer science and 
biology is essential to the further advancement of
both basic genome biology and high-level clinical applications of this new knowledge.

The pace at which new technologies and data acquisition methods emerge
makes computational and statistical genomics an extremely dynamic field.  
It is our goal to bring wet-lab biologists and research-oriented clinicians with interest in
computational biology and statisticians working in several aspects of
statistical genomics together in this workshop. 
This would serve as a great opportunity 
(1) to summarize new advances in biological
technologies and state of the art statistical methodologies addressing
relevant challenges, 
(2) to criticize and discuss limitations of the
existing methodologies and formulations,
(3) to explore ways to solve
these issues, and 
(4) to discuss areas where more interaction among
the two communities is needed.

We list below several areas of biological investigation that are
fueled by technological advances and require rigorous statistical and
computational analysis. 
There are no strict borders between topics,
since most share high dimensional multivariate data that are
similar in nature, and biological discoveries are often achieved
through merging of various sources of data and perspectives. 
The workshop will focus around these five topics.  
For each topic, related statistical aspects such as 
parameter specification, estimation,
inference and testing, model selection, and statistical computing
issues will be addressed. 
We aim to have at least one well-known plenary biologist 
and one statistician to speak on each topic.  
Aside from regular talks, poster and software 
demonstration sessions will provide researchers opportunities
to present current applications and results on these topics.  

We have pre-invited some of the possible speakers from both
communities including 

***should we specify the main area of work for each?****
***how many we should pre-invite and what if they don't reply?***
**we will need to cut ... since i am adding clinicians/large scale biologists for the translational part of the workshop, we will need to cut down on basic biologists***
***do we need to focus on BIG names for pre-invitations (i think so) (and move smaller names to the list below or it does not matter?***

Mauro Delorenzi (Swiss Experimental Cancer Research Institute/Swiss Institute of Bioinformatics),
Wolfgang Huber (EBI-EMBL)
Simon Tavare (USC and Cambridge University, UK),
Natalie Thorne (ambridge University, UK),
Yee Hwa Yang (University of Sydney, Australia),
Christos Sotiriou (Jules Bordet Institut, Brussels, Belgium) ***clinical, breast cancer,
Pratyaksha Wirapati (Swiss Experimental Cancer Research Institute/Swiss Institute of Bioinformatics),
Ruth Luthi-Carter (EPFL, Lausanne, Switzerland) ***biologist.
Joe Gray (LBL) ** biologist/data generation***
Gordon Mills (MDA) *** clinican/biologist***
Keith Baggerly (MDA) ***statistician***
Arul Chinnaiyan (University of Michigan) ***biologistt/clinician/bioinformatics***

*** old names:  Terry Speed (University of California,
Berkeley), John Quackenbush (TIGR), David M. Rocke (University of
California, Davis), Wyeth Wasserman (University of British Columbia,
Vancouver),
Tim Hughes (University of Toronto), Rafael Irizarry (Johns Hopkins), Karl Broman (John Hopkins)
Robert Gentleman (Harvard University), Jason Lieb (University of North
Carolina, Chapel Hill), Todd Lowe (University of California, Santa
Cruz), Michael Newton (University of Wisconsin, Madison), Hao Li (University of California, San Francisco)
 and David Hinds (Perlegen  Sciences). 
All of these researchers shared our enthusiasm in such a workshop and

showed great interest in participating.



\begin{itemize}
%*** DRG - removed
%\item \textit{Phylogenomics.} Phylogenomics is a new emerging field
%that combines \textit{genomics} and \textit{molecular phylogenetics},
%which are two major fields in the life sciences.  Completion of
%whole genome sequencing projects provides scientists with a unique
%opportunity to study the origin and evolution of genomes and
%facilitates improvement of functional predictions for uncharacterized
%genes by evolutionary analysis.
%Relevant statistical research includes statistical models
%for evolution, construction and estimation of evolutionary trees,
%confidence sets of trees, and statistical models for sequence
%alignment.


%%%%%%%%%%%%%%%%%%%%%%%%%%%%%%%%%%%%%%%%%%%%%%%%%%%%%%%%%%%%%%%%
%%%
%%% 1 - Classification
%%%
%%%%%%%%%%%%%%%%%%%%%%%%%%%%%%%%%%%%%%%%%%%%%%%%%%%%%%%%%%%%%%%%


\item \textit{Classification.}

Translational aims are of paramount important in today's biomedical research. This year NCI has awarded a number of grants to begin generating atlas of genomic and genetics features in a variety of cancers. While the ultimate aim is to improve current management of cancer patients, the statistical problem is two fold. The first question is to understand how the high-dimensional data generated on the patients can be used to predict their response to standard or experimental therapies, estimate time to recurrence, assign probability of the progression under ``wait and see'' protocols as in prostate cancer or simply predict if an individual will develop a particular cancer. This problem falls into the class of {\em prediction statistical appoaches}. The second question involves identification of druggable markers of response to treatment, recurrence and progression and of early detection. This is known as {\em variable selection problem} and could lead to selecting the combinations of markers rather than indvidual ones. Note that the two questions are tightly linked. 

The statistical challenges include but are not limited to  study design and building predictors based on heterogeneous available cohorts, having relatively small ratio of sample size (in 100's, possibly in low 1000's by the time of the workshop) to the number of variables (100's of thousands of features at the moment for any given technology), multiple testing issues, development of computationally efficient classifiers able to explore interaction space of the variables and deal with the variety of the data types.


%%%%%%%%%%%%%%%%%%%%%%%%%%%%%%%%%%%%%%%%%%%%%%%%%%%%%%%%%%%%%%%%
%%%
%%% 2 - Computational population genetics 
%%%
%%%%%%%%%%%%%%%%%%%%%%%%%%%%%%%%%%%%%%%%%%%%%%%%%%%%%%%%%%%%%%%%

***should this be combined with the Classification aim?****

\item \textit{Computational population genetics.}
*** NOT CHANGED FROM PREVIOUS VERSION (should be changed???) ***
Single nucleotide polymorphism (SNPs) are the most simple form and
most prevalent source of genetic polymorphism in the human genome.
The advent of SNP genotyping and haplotyping technologies are leading
to accumulation of massive amounts of SNP data spanning a variety of
species. One of the challenges faced by researchers in this field is
how to relate such multimillion dimensional genotypic profiles to both
biological and clinical phenotypes,  such as disease and drug
reaction. As more information accumulates, analysis of the emerging
complex data requires comprehensive statistical methodologies capable
of dealing with challenging issues such as censoring and causality.



%%%%%%%%%%%%%%%%%%%%%%%%%%%%%%%%%%%%%%%%%%%%%%%%%%%%%%%%%%%%%%%%
%%%
%%% 3 - Pharmacogenomics
%%%
%%%%%%%%%%%%%%%%%%%%%%%%%%%%%%%%%%%%%%%%%%%%%%%%%%%%%%%%%%%%%%%%


\item \textit{Pharmacogenomics.}
Genome-scale data are at the forefront of research into 
targeted therapeutics/individualized medicine.
Pharmacogenomics deals with the influence of genetic variation on drug response in patients,
and is overturning the `one size fits all' paradigm of drug development and treatment.
Better understanding of an individual's genetic makeup may be a key 
element of the therapeutic regime indicated for that particular individual.
This multi-disciplinary field combines
traditional pharmaceutical sciences 
with large scale data and meta-data on genes, proteins, 
and single nucleotide polymorphisms.

However, data analysis can be difficult due to limitations in 
the present state of knowledge regarding the relevant 
signaling pathways, as well as to high noise levels inherent in such data. 
New statistical developments here have the potential to play an
important role in further progress toward individualized medicine.


%*** DRG - removed
%\item \textit{Comparative genomics.} Comparison of genomes between
%species aids in every step of the genomic analysis. Some of the areas that gained
%attention are identification of the differences between related
%genomes including presence and absence of genes and pathways, and
%regulatory sequence signals. By incorporating multiple species
%sequence data with other sources of high-throughput genomic data,
%scientists are trying to understand how sequence features control the
%activities of genes and how these features are organized into
%modules. Evolutionary conservation of regulatory modules and gene
%expression are also among aspects that can assist us in understanding
%gene function and regulatory pathways.


%%%%%%%%%%%%%%%%%%%%%%%%%%%%%%%%%%%%%%%%%%%%%%%%%%%%%%%%%%%%%%%%
%%%
%%% 4 - Emerging technologies
%%%
%%%%%%%%%%%%%%%%%%%%%%%%%%%%%%%%%%%%%%%%%%%%%%%%%%%%%%%%%%%%%%%%


%*** DRG - modified
\item \textit{Emerging technologies.}  
There are several commonly used technologies for
acquiring vast amounts of genomics and proteomics data, 
with further improvements and technological advances
rapidly giving rise to newer assays.
Recent technological advances enable collection of
many different types of data at a genome-wide scale, including: 
DNA sequences, gene and protein expression measurements, splice variants, methylation information,  
protein-protein interactions, protein structural information, 
and protein-DNA binding data. 
These data have the potential to elucidate cellular 
organization and function. 
Studies of disease processes in humans often include as additional
information various types of patient clinical data and covariates.

Each technology involves computational, mathematical, 
and statistical issues regarding data acquisition, 
processing, analysis and subsequent interpretation.
Statisticians have already contributed immensely in 
improving design and analysis of gene expression microarrays.
Similar challenges are inevitably arising
for newer platforms, such as the SNP chips used for high-throughput
genome sequencing.
Continued interdisciplinary research between biologists and quantitative
scientists is crucial to  
achieving a high level of methodological success
for analyzing these newer data types, 
which will only gain in importance.

%%%%%%%%%%%%%%%%%%%%%%%%%%%%%%%%%%%%%%%%%%%%%%%%%%%%%%%%%%%%%%%%
%%%
%%% 5 - Data integration
%%%
%%%%%%%%%%%%%%%%%%%%%%%%%%%%%%%%%%%%%%%%%%%%%%%%%%%%%%%%%%%%%%%%


\item \textit{Data integration.}  
The explosion of data is generating a number
of new challenges in statistics, mathematics and computing.
The data are of heterogeneous types measured across a number of biological organisms, very high dimensional,
and typically exhibit substantial variability in addition
to varying degrees of incompleteness.
In order to use the abundance of information
to significantly advance biological understanding, 
fundamentally sound quantitative methods for combination
of the manifold data types are required.
Appropriate data integration gives researchers power
to uncover meaningful biological relationships, 
enabling further understanding, targeted follow-up, and
efficient use of resources.

Results and findings jointly learned 
from multiple, diverse data types are likely to lead 
to new insights that are not as readily discovered by the 
analysis of just one type of data. 
So far computationally straighforward, mainly correlative approaches have been applied in 
gene expression  and, lately, copy number data, for combining study results. 
It seems clear, though, that meta-analysis is not very
straightforwardly applied to the problem of 
combining data of different types,
the most obvious impediment being lack of a 
common parameter across data types and mix of letter-based (sequence, ) categorical (SNP), ordinal (methylation, protein expression) and continuous (expression and copy number) data types.
More sophisticated approaches
include phylogenetic methods, hierarchical Bayesian models as well as variations on correlation-based approaches such as kernel, svd-type or
distance-based methods. 
However, integrating multiple data types in an
automated, quantitative manner remains a major challenge,
where innovative approaches appear to be required. Note that the challenges are so novel that they include both identifying the biologically relevant questions arising from data integration as well as specifying the statistical models and corresponding parameters for estimation along with the required statistical methodologies. 

\end{itemize}

\section*{A list of possible participants and their affiliation}

***should we specify the main area of work for each?****
***i assume we are not listing organizers, right?***

\begin{enumerate}

\item Joe Felsenstein, University of Washington.

\item Adam Seipel$\mathbf{*}$, University of California, Santa Cruz.

\item Bret Larget, University of Wisconsin, Madison.

\item Lior Pachter, University of California, Berkeley.

\item Leonid Kruglyak, University of Washington, Fred Hutchinson.

\item Charles Kooperberg, University of Washington, Fred
Hutchinson.

\item David Hinds, Perlegen Sciences.

\item Jurg Ott, Rockefeller University.

\item Hao Li, University of California, San Francisco

\item Michael Eisen, University of California, Berkeley, LBL.

\item Wing Wong, Harvard University.

\item Todd Lowe$\mathbf{*}$, University of California, Santa Cruz.

\item Jun Liu, Harvard University.

\item Tim Hughes$\mathbf{*}$, University of Toronto.

\item Jason Lieb$\mathbf{*}$, University of North Carolina, Chapel Hill.

\item Joe Derisi, University of California, San Francisco.

\item Terry Speed$\mathbf{*}$, University of California, Berkeley.

\item Rafael Irizarry$\mathbf{*}$, Johns Hopkins University.

\item Simon Cawley, Affymetrix.

\item David M. Rocke$\mathbf{*}$, University of California, Davis.

\item John Quackenbush$\mathbf{*}$, TIGR.

\item Aad van der Vaart, Vrije Universiteit.

\item Michael A. Newton, University of Wisconsin, Madison.

\item Neil Clarke, Johns Hopkins University.

\item Pat Brown, Stanford University.

\item Wyeth Wasserman$\mathbf{*}$, University of  British Columbia, 
Vancouver.

\item Steven Brenner, University of California, Berkeley.

\item Andrej Sali, University of California, San Francisco.

\item David Baker, University of Washington.

\item Ingo Ruczinski, Johns Hopkins University.

\item Robert Gentleman$\mathbf{*}$, Harvard University.

\item Tim Hubbard, The Wellcome 
Trust  Sanger Institute.

\item Rebecka J\"ornsten, Rutgers University.

\item Mark Segal, University of California, San Francisco.

\item Ry Fang Yeh, University of California, San Francisco.

\item Joe Gray, Lawrence Berkeley Laboratory

\item Paul Spellman, Lawrence Berkeley Laboratory

\item Keith Baggerly, MD Anderson

\item Jeffrey Morris, MD Anderson

\item Gordon Mills, MD Anderson

\item Karl Broman, John Hopkins

\item Shane Jensen, University of Pennsilvania

\item Adam Olshen, Sloan-Kettering

\item Natalie Thorne, Cambridge University, UK

\item Mark Van Der Laan, UC Berkeley

\item Hongyu Zhao, Yale University

\item Annette Molinaro, Yale University

\item Arul Chinnaiyan{\mathbf{*}}, University of Michigan

\item Debashishis Gnosh, University of Michigan


\end{enumerate}


\section*{Additional comments}

Our experience in organizing fruitful workshops in the 
general area of statistical genomics includes:

\begin{itemize}
\item \textit{Statistics in Functional Genomics, Ascona (Switzerland) 2004.}
This meeting, organized by Darlene Goldstein, Peter B\"uhlmann and Anthony Davison,
included about 80 biologists and quantitative scientists (30\% women) from all
over the world.
We were able to attract as invited speakers top experts from 
Berkeley, Harvard, Imperial College, and the Max Planck Institut (Berlin).
The workshop was a great success, giving rise to some
new collaborations, and was internationally very visible.

\item \textit{Statistical Science for Genome Biology, BIRS 2004.}
This 5-day workshop, organized by Jennifer Bryan, Sandrine Dudoit and
Mark J. van der Laan, brought together
statisticians working in different genomics-related aspects of statistics. 
The workshop was a huge success and covered several
broad areas of biological investigation that relied on statistical
and computational methods. 
An outstanding aspect of the workshop
was participation of many young researchers and women:  23 of the
39 participants were in the category of graduate
student/postdoctoral researcher/assistant professor. 
Moreover, 13 among these were women. 

\item \textit{Computational and Statistical Genomics, BIRS 2006.}
Another 5-day workshop, organized by Jennifer Bryan, Sandrine Dudoit, 
Sunduz Keles, Katherine S. Pollard and Mark van der Laan, 
addressed computational and statistical problems
associated with newer data types.
This profitable meeting included a wide range of expertise,
a large proportion of it from women.

\item \textit{Statistics for Biomolecular Data Integration and Modeling, Ascona 2007.}
This workshop, with a focus on combining data and systems biology, is again organized by Darlene Goldstein, Peter B\"uhlmann and Anthony Davison.
It is scheduled to take place in June 2007.
\end{itemize}

We would like to build on our track record of these extremely
successful workshops, while adapting the scope to new emerging fields of
statistical genomics. 
Enabling close communication between the
biological and statistical communities
will be conducive to creating statistical innovations that are
relevant for advancement in the new and challenging
problems that we plan to address.


\section*{Dates}

Our preferred week (July 27-31) would be particularly
advantageous, since the Joint Statistical Meetings (the largest
statistics conference in North America) will be held in Denver, Colorado
from August 3 - 7, 2008.

\subsection*{Preferred dates}

\begin{itemize}
\item July 27 - 31, 2008.
\item June 1 - 5, 2008.
\item June 22 - 26, 2008.
\item June 29 - July 1, 2008.
\item July 6 - 10, 2008.
\end{itemize}

\subsection*{Impossible dates}

\begin{itemize}
\item ????, 2008; ISMB. **could not find dates but small enough conference that should not really matter. remove?***
\item June 13 - 19, 2008; IMS.
\item August 3 - 7, 2008; JSM.
\item July 13 - 18, 2008; IBS.
\item ???, 2008; WNAR. **could not find dates but small enough conference that should not really matter. remove?***
\item March 16 - 19, 2006; ENAR.
\end{itemize}

\end{document}

