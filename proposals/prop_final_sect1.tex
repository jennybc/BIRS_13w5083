\documentclass{amsart}

\begin{document}

\title{Proposed workshop ``Statistical Science for Genome Biology''}

\maketitle

\section*{Subject Area Overview}

Recent developments in genetics and molecular biology have unleashed a
flood of data that is noteworthy for its volume, complexity, and
noisiness. Some of the most significant developments of the past
decade in the biological sciences have relied on successfully
integrating biological expertise with statistical thinking to design
and analyze genomic experiments. While the connections between the
biological and mathematical sciences are by no means new, as
exemplified by the early genetic mapping work of the statistician
R. A. Fisher in the 1920s, interactions between the two fields are
more crucial and extensive than ever.  Indeed, as noted by Eric
S. Lander and Robert A. Weinberg in "Genomics: Journey to the Center
of Biology" (Science 2000 March 10; 287: 1777-1782): "The availability
of the complete parts lists of organisms, that is, catalogs of all
their genes and thus all their proteins, has been redirecting
biologists toward a global perspective on life processes -- to study
the role of all genes or all proteins at once. ... This new approach
promises stunning breadth of perspective. At the same time, it
threatens to inundate scientists in a flood of data that will
overwhelm their ability to interpret it. Powerful new types of
bioinformatics will clearly be required to assimilate and interpret
the data that will issue from various types of genomic research."

We highlight here several broad areas of biological investigation
which rely vitally on statistical and computational methods for
assembling, processing, and interpreting large multivariate datasets:
\begin{itemize}
\item {\em Sequence analysis, for both nucleic acids and proteins.}
The Human Genome Project is the most striking illustration of the
power of current high throughput sequencing capabilities. Genomic
sequences from multiple and diverse organisms are now available for
comparative analyses with the aim of understanding the structure,
function, and evolution of genomes. Specific problems in nucleotide
sequence analysis include: whole genome assembly, sequence alignment,
computational gene finding, identification of regulatory
regions. Protein folding and structure prediction are of great
interest in drug discovery, among other areas.
\item {\em Genetic mapping.}  Small-scale sequencing efforts are
underway to catalog hundreds of thousands of single nucleotide
polymorphisms (SNPs) and describe their biological significance . A
promising approach toward understanding the genetic basis of complex
diseases, such as diabetes or multiple sclerosis, involves detecting
associations between the trait and SNP haplotypes (i.e., combinations
of alleles at closely linked SNP markers).
\item {\em Gene expression profiling.}  DNA microarrays allow the
measurement of messenger RNA abundance in cells for thousands of genes
simultaneously.  Similar methods for proteins are in active
development. Such large-scale gene expression assays are increasingly
being performed in biological and medical research to address a wide
range of problems.  In cancer research, for example, DNA microarrays
are used to study the molecular variations among tumors with the aim
of developing better diagnosis and treatment strategies for the
disease.
\item {\em Molecular interaction.}  Immunoprecipitation and yeast
two-hybrid studies are just two examples of methods used to determine
whether individual molecules, in this case proteins, interact with
each other.
\end{itemize}
The above are four examples of genomic research areas which absolutely
require quantitative analysis in order to translate the observed
measurements into biological knowledge. Furthermore, the underlying
biological variation and unavoidable measurement errors imply that
statistical formulations are universally applicable. Although existing
statistical paradigms are extremely powerful for analyzing genomic
data, certain aspects of these data present new challenges and,
therefore, motivate new research.  Areas of statistical research
include:
\begin{itemize}
\item {\em Complex parameter specification and estimation.}  The
actual parameter of interest in genomic studies is rarely explicitly
identified; more often, it is implicit in the research goals. We are
generally trying to recover information about extremely complex
objects, such as gene networks.  Even simplified versions of this,
such as identifying members of a network in a gene list or sorting
members of distinct networks into gene clusters, implicitly define
very high-dimensional parameters.  Although it is rarely done in
practice today, the estimates produced in real-world data analysis
should be accompanied by relevant measures of uncertainty.
Statisticians are developing appropriate notions of sampling error
(what is the 'standard error' of an estimated gene network or of a
tree-based gene clustering?) and valid methods for estimating that
error.
\item {\em Multiple testing.}  The ``multiplicity problem'' arises
when a large number of hypothesis tests are being performed at once
(or a large number of interval estimates are being simultaneously
constructed).  Since the dimension of the parameter of interest in a
genomic study is typically driven by the number of genes in a genome,
multiplicity problems are encountered on an unprecedented scale.
Statisticians are actively devising ways to control the probabilistic
performance of testing and estimation procedures for high-dimensional
parameters, in the presence of large errors and correlated data.
\item {\em Cluster analysis, prediction, and model selection.}
Sequencing, genotyping, and gene expression projects produce
measurements on thousands of genomic variables. It is of interest to
relate these variables to other biological and clinical variables to
elucidate, for example, the molecular basis of complex diseases such
as cancer. In cancer research, large databases of clinical, sequence,
microarray, and SNP data are being assembled. Goals include the use of
genomic variables to distinguish among known tumor classes, to predict
clinical outcomes such as survival and response to treatment, and to
identify previously unrecognized and clinically significant subclasses
of tumors. The development of clustering, prediction, and model
selection methods for high-dimensional problems is critical for these
efforts.  Assessing the reliability of the inferences is a key aspect
for each of these tasks.
\item {\em Statistical computing.} An efficient computing environment
is essential for the analysis of genomic data. Such an environment
should provide access to a broad range of statistical and graphical
methods, as well as tools for integrating biological metadata
(e.g. textual data from PubMed, annotation data from the Gene Ontology
(GO) Consortium) in the analysis of experimental data. Novel software
design and distributed computing paradigms are being investigated.  
\end{itemize}
In addition to conducting research on new methodologies, statisticians
working in genomics must also identify existing techniques that are
appropriate for these data structures and demonstrate how they can be
correctly applied in new settings.  Bringing novel and existing
methods to bear on genomic data, statisticians can have tremendous
positive impact on the knowledge derived from genome-scale
experiments.

\end{document}

