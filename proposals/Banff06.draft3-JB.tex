\documentclass[12pt]{amsart}
\oddsidemargin  -0.3in
\evensidemargin -0.3in
\textwidth 7in
\headheight     0in
\textheight=8.5in
\topmargin -0.5in
\headsep 0.5in

\usepackage{color}
%\definecolor{purple}{rgb}{0.65, 0, 0.75}

\newcounter{mynote} \newenvironment{mynote}{\stepcounter{mynote}
\color{red} }{}

\newcommand{\comma}{\marginpar {$\Leftarrow$}}

\title{PROPOSED WORKSHOP ``COMPUTATIONAL AND STATISTICAL GENOMICS''}
\begin{document}
\maketitle

\begin{mynote}
Jenny's edits appear in red.  Arrows appear in the margin when I've
inserted a comma, which, I, think, we, need, more, of! \comma
\end{mynote}

\section*{Objectives}
The main objective of this workshop is
\begin{mynote}
to formulate and address important statistical problems
\end{mynote}
in the analysis of biomedical and high-throughput genomic data, \comma
including DNA chip, ChIP-chip, SNP, whole-genome sequence and
proteomic data. A distinct goal is to facilitate the interaction
between biologists performing genome scale experiments (``wet-lab''
researchers) and statisticians
\begin{mynote}
with expertise
\end{mynote}
and interest in genomics (``dry-lab'' researchers).
\begin{mynote}
Substantive collaborations between wet and dry lab scientists are
vital for transforming the massive amount of data produced by new
technologies into important biological discoveries.
\end{mynote}

%More specifically, the objective of the meeting will be to promote the current development in
%statistical genomics by bringing together wet lab computational biologists  and statisticians with
%interest in  genomics.

The workshop is intended to foster deeper connections between the two
research communities and to be a forum for 
\begin{mynote}
(1) the dissemination of cutting-edge developments, including new
high-throughput biological assays and novel statistical methodologies
and (2) the identification of open problems in the analysis of these
data.
\end{mynote}
These challenges
\begin{mynote}
include not only analyzing
\end{mynote}
genotypic data, \comma but also relating these to phenotypic data,
\comma such as biological and clinical outcomes, and further relating
both to meta-data from WWW databases, such as PubMed and Gene Ontology
(GO) Consortium.

We anticipate that this workshop will enable statisticians to
\begin{mynote}
articulate
\end{mynote}
theoretically grounded statistical formulations of existing and
emerging computational biology problems; create an exceptional
opportunity for exchanging ideas between the communities; and help to
shape the future of this dynamic field. Input from wet-lab
computational biologists is absolutely crucial for development of
appropriate statistical methodologies. For this reason, we have
targeted areas that
\begin{mynote}
are relatively
\end{mynote}
new to statisticians, \comma as well as areas
\begin{mynote}
that have already been greatly influenced by statistical approaches.
\end{mynote}
These include phylogenomics, computational population genetics,
comparative genomics, microarray technologies and protein structure
prediction. We anticipate that the interaction between statisticians
and wet-lab computational biologists will lead to major advances in
computational and statistical genomics.

\section*{Relevance, importance and timeliness} It is now well
accepted that the capacity to generate genome data has far outpaced
our ability to analyze and interpret it.
\begin{mynote}
The rapid development of new and existing high-throughput technologies
is allowing biologists to investigate biological processes on an ever-
growing scale.
\end{mynote}
Statistical genomics adapts well to these changes, \comma due to the
incredible interest of statisticians
\begin{mynote}
in these methodological challenges.  Since this is a relatively new
and rapidly developing field, it enjoys an above-average
representation of young talent and women.  Rapid communication between
wet-lab computational biologists and statisticians is absolutely
vital.  While there are several well-established computational biology
conferences (e.g., ISMB and RECOMB), the primary quantitative
discipline has historically been computer science, not statistical
science.
\end{mynote}
Likewise, even though major statistical conferences often have
sessions on computational biology, the number of those are too few,
and the audience is almost exclusively statisticians.  Hence, a
workshop that specifically brings wet-lab biologists and statisticians
together is sorely needed and BIRS provides an unbeatable environment
for this task.

\newpage

\section*{Subject area overview}
\begin{mynote}
Modern high-throughput technologies are changing the face of
biomedical and life science research.
\end{mynote}
Today, researchers investigating a molecule or process in any given
organism (including the human) often have the complete DNA sequence of
that organism.
\begin{mynote}
However, determining
\end{mynote}
the DNA sequence is just the first step in understanding the structure
of a genome and the functions of its genes.
\begin{mynote}
Where labs used to focus on single genes and proteins, they now aim to
integrate vast amounts of ever-changing types of data to study
complicated entities, such as protein complexes and regulatory
networks.  The dawning of the ``post-genomic'' era of biology
\end{mynote}
requires an interdisciplinary approach. Cooperations among statistics,
mathematics, computer science and biology are vital to the future of
genome biology. A recent news feature article in Nature (August 7,
2003) emphasizes the
\begin{mynote}
importance of sound statistical analysis of genomic data.  In
reference to the ``growing number of statistical experts [familiar
with] \dots the complexities of microarrays'', the article concludes
with some advice: ``In the meantime,the message to biologists is
clear: if you want to work with microarrays, you need to find yourself
one of these precious experts and don't wait until after you've
collected your data.''
\end{mynote}
\marginpar {Better quote?} This observation is even more true for
other areas of computational and statistical genomics, \comma such as
phylogenomics, protein structure prediction, computational population
genetics, and comparative genomics which we highlight later in this
section.

The pace at which new technologies and data acquisition methods emerge
makes computational and statistical genomics an extremely dynamic
field.  It is our goal to bring wet-lab biologists with interest in
computational biology and statisticians working in several aspects of
statistical genomics together in this workshop. This would serve as a
great opportunity (1) to summarize new advances in biological
technologies and state of the art statistical methodologies addressing
relevant challenges, (2) to criticize and discuss limitations of the
existing methodologies and formulations, (3) to explore ways to solve
these issues, and (4) to discuss areas where more interaction among
the two communities
\begin{mynote}
is needed.
\end{mynote}

We list below several areas of biological investigation that are
fueled by technological advances and require rigorous statistical and
computational analysis. There are no strict borders between topics,
\comma since most share high dimensional multivariate data that are
similar in nature and biological discoveries are often achieved
through merging of various sources of data and perspectives. The
workshop will focus around these five topics.  For each topic, related
statistical problems such as parameter specification, estimation,
inference and testing, model selection, and statistical computing
issues will be addressed. We are aiming to have at least one
well-known plenary wet-lab computational biologist and one
statistician to speak on each of these topics.  Aside from regular
talks, poster and software demonstration sessions will provide
researchers to present current applications and results on these
topics.  We have pre-invited some of the possible speakers from both
communities including Terry Speed (University of California,
Berkeley), John Quackenbush (TIGR), David M. Rocke (University of
California, Davis), Wyeth Wasserman (University of British Columbia,
\begin{mynote}
Vancouver),
\end{mynote}
Tim Hughes (University of Toronto), Rafael Irizarry (Johns Hopkins),
Robert Gentleman (Harvard University), Jason Lieb (University of North
Carolina, Chapel Hill), Todd Lowe (University of California, Santa
Cruz), and David Hinds (Perlegen \begin{mynote}
Sciences). 
\end{mynote}
All of these researchers shared our enthusiasm in such a workshop and
showed great interest in participating.

\begin{mynote}
Can we really do anything about this now? Our pre-invitation list and
our longer list of participants does not have nearly the female
representation we enjoyed in 2004.
\end{mynote}

\begin{itemize}
\item \textit{Phylogenomics.} Phylogenomics is a new emerging field
that combines \textit{genomics} and \textit{molecular phylogenetics},
\comma which are two major fields in the life sciences.  Completion of
whole genome sequencing projects provides scientists with a unique
opportunity to study the origin and evolution of genomes and
facilitates improvement of functional predictions for uncharacterized
genes by evolutionary analysis.
\begin{mynote}
Relevant statistical research includes statistical models
\end{mynote}
for evolution, construction and estimation of evolutionary trees,
confidence sets of trees, and statistical models for sequence
alignment.

%\textit{Invitees.} Joe Felsenstein (University of Washington);
%Adam Seipel (University of California, Santa Cruz);
%Bret Larget (University of Wisconsin, Madison); Lior Pachter (University of California, Berkeley).\\


\item \textit{Computational population genetics.}
%Determining the
%genetic basis of disease susceptibility has long been an ultimate
%goal of geneticists.
Single nucleotide polymorphism (SNPs) are the most simple form and
most prevalent source of genetic polymorphism in the human genome.
The advent of SNP genotyping and haplotyping technologies are leading
to accumulation of massive amounts of SNP data spanning a variety of
species. One of the challenges faced by researchers in this field is
how to relate such multimillion dimensional genotypic profiles to both
biological and clinical phenotypes, \comma such as disease and drug
reaction. As more information accumulates, analysis of the emerging
complex data requires comprehensive statistical methodologies capable
of dealing with challenging issues such as censoring and causality.


%Parametrization and estimation of haplotype
%structures on different chromosomes and relating these to
%phenotypes such as disease and drug reaction, constructing
%haplotype trees  are among the challenging computational and
%statistical questions scientists strive to answer using these vast
%amounts of data.

%\textit{Invitees.} Leonid Kruglyak (University of Washington, Fred
%Hutchinson); Charles Kooperberg
%(University of Washington, Fred Hutchinson); David Hinds (Perlegen Sciences);
%Jurg Ott (Rockefeller University).\\


\item \textit{Comparative genomics.} Comparison of genomes between
species
\begin{mynote}
aids
\end{mynote}
in every step of the genomic analysis. Some of the areas that gained
attention are identification of the differences between related
genomes including presence and absence of genes and pathways, and
regulatory sequence signals. By incorporating multiple species
sequence data with other sources of high-throughput genomic data,
scientists are trying to understand how sequence features control the
activities of genes and how these features are organized into
modules. Evolutionary conservation of regulatory modules and gene
expression are also among aspects that can assist us in understanding
gene function and regulatory pathways.

%\textit{Invitees.} Hao Li (University of California, San
%Francisco); Michael Eisen (University of California, Berkeley);
%Wing Wong (Harvard University); Todd Lowe (University of
%California, Santa Cruz); Jun Liu (Harvard University) .\\


\item \textit{Microarray technologies.}  The types of biological
investigations that microarrays enable are increasing rapidly.  Today,
microarrays are used in areas from gene expression profiling to
protein expression profiling and whole-genome profiling of
interactions between DNA-binding proteins and DNA.  Statisticians have
already contributed immensely in improving design and analysis of gene
expression microarrays and similar challenges are inevitably arising
for other platforms, such as the SNP chips used for high-throughput
genome sequencing, that utilize microarray technology.

%\textit{Invitees.} Tim Hughes (University of Toronto); Jason Lieb
%(University of North Carolina, Chapel Hill); Joe Derisi
%(University of California, San Francisco); Mark J. van der Laan
%(University of California, Berkeley); Terry Speed (University of
%California, Berkeley); Rafael Irizarry (Johns
%Hopkins); Simon Cawley (Affymetrix); David M. Rocke (University of California, Davis).\\

\item \textit{Protein structure prediction.} A crucial step in
understanding of human biology is the characterization of all human
proteins. This requires the knowledge of their three dimensional
structures since structural information leads to protein function and
aids in drug design.
%One critical aspect of structure determination is predicting the effects of sequence 
%variations on these structures. 
Protein structure prediction is one of the more elusive goals of
computational biology where rigorous modern statistical techniques
such as prediction tools for high-dimensional data can provide
improvement and new perspectives.  Up to date, there are relatively
few statisticians involved in this area, thus exposure of more
statisticians to this challenging and exciting field is essential.

\end{itemize}

\section*{A list of possible participants and their affiliation}
\begin{enumerate}
\item Joe Felsenstein, University of Washington.

\item Adam Seipel$\mathbf{*}$, University of California, Santa Cruz.

\item Bret Larget, University of Wisconsin, Madison.

\item Lior Pachter, University of California, Berkeley.

\item Leonid Kruglyak, University of Washington, Fred Hutchinson.

\item Charles Kooperberg, University of Washington, Fred
Hutchinson.

\item David Hinds, Perlegen Sciences.

\item Jurg Ott, Rockefeller University.

\item Hao Li, University of California, San Francisco

\item Michael Eisen, University of California, Berkeley, LBL.

\item Wing Wong, Harvard University.

\item Todd Lowe$\mathbf{*}$, University of California, Santa Cruz.

\item Jun Liu, Harvard University.

\item Tim Hughes$\mathbf{*}$, University of Toronto.

\item Jason Lieb$\mathbf{*}$, University of North Carolina, Chapel Hill.

\item Joe Derisi, University of California, San Francisco.

\item Terry Speed$\mathbf{*}$, University of California, Berkeley.

\item Rafael Irizarry$\mathbf{*}$, Johns Hopkins.

\item Simon Cawley, Affymetrix.

\item David M. Rocke$\mathbf{*}$, University of California, Davis.

\item John Quackenbush$\mathbf{*}$, TIGR.

\item Aad van der Vaart, Vrije Universiteit.

\item Michael A. Newton, University of Wisconsin, Madison.

\item Heping Zhang, Yale University.

\item Hongyu Zhao, Yale University.

\item Neil Clarke, Johns Hopkins University.

\item Pat Brown, Stanford University.

\item Wyeth Wasserman$\mathbf{*}$, University of  British Columbia, \begin{mynote}
Vancouver.
\end{mynote}

\item Steven Brenner, University of California, Berkeley.

\item Andrej Sali, University of California, San Francisco.

\item David Baker, University of Washington.

\item Ingo Ruczinski, Johns Hopkins University.

\item Robert Gentleman$\mathbf{*}$, Harvard University.

\item Tim Hubbard, The Wellcome \begin{mynote}
Trust 
\end{mynote}
Sanger Institute.
\end{enumerate}



\section*{Additional comments}

A 5-day workshop, organized by Jennifer Bryan, Sandrine Dudoit and
Mark J. van der Laan in 2004 at BIRS, brought together
statisticians working in different genomics related aspects of
statistics. The workshop was a huge success and covered several
broad areas of biological investigation that relied on statistical
and computational methods. An outstanding aspect of the workshop
was participation of many young researchers and women. 23 of the
39 participants were in the category of graduate
student/postdoctoral researcher/assistant professor. Moreover, 13
among these were women. We would like to build on this extremely
successful workshop but adapt its scope to new emerging fields of
statistical genomics and extend the invitee list  to wet-lab
computational biologists to enable close communications between
two communities.

\begin{mynote}
Our preferred week (August 13 - 17) would be particularly
advantageous, since the Joint Statistical Meetings (the largest
statistics conference in North America) will be concluding in Seattle,
WA on August 11.
\end{mynote}


\section*{Dates}

\subsection*{Preferred dates}

\begin{itemize}
\item August 13 - 17, 2006.
\item June 11 - 15, 2006.
\item June 3 - 7, 2006.
\item July 9 - 13, 2006.
\item September 13 - 17 or October 8 - 12 (off-season dates).
\end{itemize}


\subsection*{Impossible dates}

\begin{itemize}
\item May    28 - 31, 2006; Statistical Society of Canada.
\item August 6 - 11, 2006; ISMB.
\item July   30 - August 4, 2006; IMS.
\item August 6 - 10, 2006; JSM.
\item Late June, 2006; WNAR.
\item March 19 - 22, 2006; ENAR.
\end{itemize}

\end{document}

