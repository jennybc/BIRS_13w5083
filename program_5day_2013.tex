%%%%%%%%%%%%%%%%%%%%%%%%%%%%%%%%%%%%%%%%%%%%%%%%%%%%%%%%%%%%%%%%
%
% Last modified: Jan 22, 2013
% Template latex file to create a schedule and abstracts
% for BIRS 5 day workshops in 2013--.
%
% 1. Use the command: 
%        pdflatex programme.tex
%    to create the pdf file programme.pdf.
% 
% 2. Email programme.pdf to BIRS Station Manager birsmgr@birs.ca.
%
% 3. Questions and comments should be sent to either 
%    birs@birs.ca or birsmgr@birs.ca.
%
% WORKSHOP SCHEDULING ADDITIONAL GUIDELINES:
% 
% 1. LECTURES:  
% 
% You are welcome to set up your own schedule for lectures at BIRS.  The
% attached recommended schedule is just a guideline.  The following are a
% few points to consider should you decide to create your own schedule. 
% (a)  Many BIRS participants have commented that having no more than 4-5
% lectures per day is satisfactory.  Some groups have opted to work
% through the day and then have evenings free while other groups have
% opted to work the mornings and evenings. 
% (b)  Meal times cannot be changed and are as follows:  Breakfast, 7-9
% am, Lunch, 11:30 am-1:30 pm, Dinner, 5:30-7:30 pm.  They are all served
% buffet style so you may join the meals at any time during these set
% hours.
% (c)  It is encouraged that you leave one afternoon completely free so
% that participants can have the time to explore the area of Banff and
% Lake Louise.  There are many types of extracurricular activities one can
% enjoy at any time throughout the year.
% 
% 2. COFFEE BREAKS:
% 
% When you are setting up your schedule, please keep in mind that your
% coffee breaks should be 20-30 minutes in length, as it takes a minute or
% two to walk from the meeting room to the coffee break room.  
% 
% Morning coffee breaks are available from 10.00 am onwards, but must finish by 11.00 am
% 
% Afternoon coffee breaks are available from 2.00 pm, but must finish by 3.30 pm.
%
% 3.  Friday:
% 
% Most participants have scheduled their departure flights throughout the
% day on Friday.  Some are forced to take early flights due to long
% journeys.  Please check with your Friday speakers to ensure that they
% are not among those who are scheduled to leave before or during their
% own talk that day.
% 
% 4.  GROUP PHOTO:
% 
% The time for the group photo (Monday, after the Banff Centre tour)
% can be changed, should you feel that another day and/or time is
% suitable for your group.  Should you decide on another time, please
% contact the Station Manager to arrange the details.  Minimum 24 hours
% notice required.
% 
% 5.  TOUR OF THE BANFF CENTRE:
% 
% We recommend that participants take part in a free guided tour of The
% Banff Centre facilities.  We suggest taking the tour on Monday (early in
% the week) so that participants can appreciate and perhaps take in some
% of the many other programs and services that are offered at The Banff
% Centre. Minimum 24 hour notice required.
%
%%%%%%%%%%%%%%%%%%%%%%%%%%%%%%%%%%%%%%%%%%%%%%%%%%%%%%%%%%%%%%%%

\documentclass[11pt]{article}

\usepackage{color}
\usepackage[pdftex]{graphicx}

\usepackage{amssymb}
\usepackage{amsmath}
\usepackage[breaklinks=true]{hyperref}

\setlength{\textwidth}{7.0in}
\setlength{\textheight}{9.5in}
\setlength{\oddsidemargin}{-0.30in}
\setlength{\evensidemargin}{0.25in}
\setlength{\topmargin}{-1in}
\definecolor{lgray}{gray}{.8}
\newcommand{\xbox}{\colorbox{lgray}{ X }}

\begin{document}

\begin{center}

\Large{\bf 13w5083: Statistical Data Integration Challenges in Computational Biology: Regulatory Networks and Personalized Medicine}\\
\Large{\bf Aug 11 - Aug 16, 2013}\\
\end{center}

\begin{center}
{\bf MEALS}
\end{center}

\noindent
*Breakfast (Buffet): 7:00--9:30 am, Sally Borden Building, Monday--Friday\\
*Lunch (Buffet): 11:30 am--1:30 pm, Sally Borden Building, Monday--Friday\\
*Dinner (Buffet): 5:30--7:30 pm, Sally Borden Building, Sunday--Thursday\\
Coffee Breaks:  As per daily schedule, in the foyer of the TransCanada Pipeline Pavilion (TCPL)\\	
{\bf *Please remember to scan your meal card at the host/hostess
station in the dining room for each meal.}

\begin{center}
{\bf MEETING ROOMS}
\end{center}

\noindent
All lectures will be held in the lecture theater in the TransCanada Pipelines Pavilion (TCPL). An LCD projector, a laptop, a document camera, and blackboards are available for presentations.

\begin{center}
{\bf SCHEDULE}
\end{center}

%\noindent
%You are welcome to schedule lectures as you see fit, as long as you adhere
%to the meal times (noted above), coffee break start and end times (noted below)
%and take into account the "welcome'' on Monday morning, the Banff Centre tour at 1:00 pm, and the group photo at 2:00 pm every Monday afternoon.\\
%
%\noindent
%Please email your finalized schedule and abstracts to BIRS Station Manager birsmgr@birs.ca by Thursday morning before your arrival (at the latest) in order to allow for printing and posting to the website.\\
%
%\noindent
%You are also encouraged to e-mail the schedule to your participants.  BIRS provides the option of an electronic mail list in order to facilitate communications with your participants.  When you login to the Organizer Interface at https://www.birs.ca/orgs, you will be prompted to create an electronic mail list for your workshop.  Click "Yes" to create one and receive instructions, or "No" to decline.  If you would like more information about our electronic mail lists, please e-mail help@birs.ca.
%\\

\begin{tabular}{ p{1in} p{5.8in} } 

{\bf\large Sunday} & \\

{\bf 16:00} & Check-in begins (Front Desk - Professional Development Centre -  open 24 hours)\\
{\bf 17:30--19:30} & Buffet Dinner, Sally Borden Building\\
{\bf 20:00} & Informal gathering in 2nd floor lounge, Corbett Hall\\
            &Beverages and a small assortment of snacks are available on a cash honor system.\\
\end{tabular}

\begin{tabular}{ p{1in} p{5.8in} } 
{\bf\large Monday} & \\

{\bf 7:00--8:45}  & Breakfast\\
{\bf 8:45--9:00}  & Introduction and Welcome by BIRS Station Manager, TCPL\\
{\bf 9:00--9:35}  & Chad Creighton, {\bf Pathway-level insights from The Cancer Genome Atlas (TCGA)}\\
{\bf 9:35--10:10}  & Yoav Gilad, {\bf Understanding gene regulation (or not)}\\
{\bf 10:10--10:45}  & Stephen Montgomery, {\bf The extent and impact of rare non-coding variants in humans}\\
{\bf 10:45--11:05} & Coffee Break, TCPL\\
{\bf 11:05--12:05}  & David Haussler, {\bf Large-scale comparative genomics for cancer research}\\
{\bf 12:05--13:00} & Lunch\\
{\bf 13:00--14:00} & Guided Tour of The Banff Centre; meet in the 2nd floor lounge, Corbett Hall\\
{\bf 14:00--14:15} & Group Photo; meet in foyer of TCPL (photograph will be taken outdoors so a jacket might be required). \\
{\bf 14:15--14:50}  & Barry Taylor, {\bf Outlier genomics drives precision oncology }\\
{\bf 14:50--15:10} & Coffee Break, TCPL\\
{\bf 15:10--16:10}  & Scott Boyd, {\bf Monitoring Human Lymphocyte Populations with High-Throughput DNA Sequencing}\\
{\bf 16:10--16:45}  &  Benjamin Haibe-Kains, {\bf Prediction of Drug Response in Cell Lines: Are Pharmacogenomic Datasets Consistent?}\\
{\bf 17:30--19:30} & Dinner\\
\end{tabular}

\begin{tabular}{ p{1in} p{5.8in} } 
{\bf\large Tuesday} & \\

{\bf 7:00--9:00}  & Breakfast\\
{\bf 9:00--9:35}  &  Pei Wang, {\bf Regularized multivariate regression approaches for integrative genomic analysis}\\
{\bf 9:35--10:10}  &  Ronglai Shen, {\bf Pattern discovery and cancer gene identification in integrated cancer genomic data}\\
{\bf 10:10--10:45}  &  Sunduz Keles, {\bf Integrative analysis of *-seq datasets for a comprehensive understanding of regulatory roles of repetitive regions}\\
{\bf 10:45--11:05} & Coffee Break, TCPL\\
{\bf 11:05--12:05}  & Hongyu Zhao, {\bf Joint analysis of expression profiles from multiple cancers to identify microRNA-gene interactions}\\
{\bf 12:05--13:25}& Lunch\\
{\bf 13:25--14:00}  &  Christina Kendziorski, {\bf Latent Dirichlet allocation models to enable personalized genomic medicine}\\
{\bf 14:00--14:35}  &  Ingo Ruczinski, {\bf Sequencing family members to detect disease risk variants}\\
{\bf 14:35--15:10}  &  Venkat Seshan, {\bf To adjust or not to adjust: the design and analysis of an epidemiologic study}\\
{\bf 15:10--15:30} & Coffee Break, TCPL\\
{\bf 15:30--16:30}  &  Colin Begg, {\bf Use of Tumor Mutational Profiles to Infer Etiologic Heterogeneity of Cancers}\\
{\bf 17:30--19:30} & Dinner\\
{\bf 20:00}  & Posters and Software Demo, TCPL\\
\end{tabular}

\begin{tabular}{ p{1in} p{5.8in} } 
{\bf\large Wednesday} & \\

{\bf 7:00--9:00}  & Breakfast\\
{\bf 9:00--9:35}  &  Pierre Neuvial, {\bf Improved performance evaluation of DNA copy number analysis methods in cancer studies}\\
{\bf 9:35--10:10}  &  Laurent Jacob, {\bf Correcting gene expression data when neither the unwanted variation nor the factor of interest are observed}\\
{\bf 10:10--10:45}  &  Roger Peng, {\bf Reproducible Research with Evidence-based Data Analysis}\\
{\bf 10:45--11:05} & Coffee Break, TCPL\\
{\bf 11:05--12:05}  &  Keith Baggerly, {\bf When is Reproducibility an Ethical Issue? Genomics, Personalized Medicine, and Human Error}\\
{\bf 12:05--13:00} & Lunch\\
{\bf 13:00--19:00} & Hiking\\
{\bf 17:30--19:30} & Dinner Served\\
{\bf 20:00} & Conference Dinner, Place TBA\\
\end{tabular}

\begin{tabular}{ p{1in} p{5.8in} } 
{\bf\large Thursday} & \\

{\bf 7:00--9:00}  & Breakfast\\
{\bf 9:00--9:35}  &  Kasper Hansen, {\bf A genome-wide look at DNA methylation}\\
{\bf 9:35--10:10}  &  Wolfgang Huber, {\bf Differential analysis of count data from high-throughput sequencing}\\
{\bf 10:10--10:45}  &  Mark Segal, {\bf Reproducibility of 3D chromatin configuration reconstructions}\\
{\bf 10:45--11:05} & Coffee Break, TCPL\\
{\bf 11:05--12:05}  & Jeff Leek, {\bf Statistical processes for facilitating personalized medicine}\\
{\bf 12:05--13:45}& Lunch\\
{\bf 13:45--14:20}  &  Anshul Kundaje, {\bf Learning long-range regulatory interactions and unified gene regulation programs in diverse human cell-types}\\
{\bf 14:20--14:55}  &  Simon Gravel, {\bf Personal genomics of the Mestizos}\\

{\bf 14:55--15:15} & Coffee Break, TCPL\\
{\bf 15:15--16:15}  &  X. Shirley Liu, {\bf Integrating sequencing and microarray data to identify novel functions of epigenetic regulators in cancer}\\
{\bf 17:30--19:30} & Dinner\\
{\bf 20:00}  & Pub evaluation led by Stephen Montgomery\\
\end{tabular}

\begin{tabular}{ p{1in} p{5.8in} } 
{\bf\large Friday} & \\

{\bf 7:00--8:30}  & Breakfast\\
{\bf 8:30--9:05} & Dave Stephens, {\bf Statistical modeling and computation for methylation profiles in the BLK gene region}\\
{\bf 9:05--9:45} & Noah Simon, {\bf Estimating Many Effect-sizes Bayesian Estimation as a Frequentist}\\
{\bf 9:45--10:10} & Alexis Battle, {\bf Characterizing the genetic basis of transcriptome diversity through RNA-sequencing}\\
{\bf 10:10--10:45} & Davide Risso, {\bf The role of spike-in standards in the normalization of RNA-Seq}\\
{\bf 11:30--13:30} & Lunch\\
\end{tabular}

\begin{tabular}{ p{1in} p{5.8in} } 
{\bf Checkout by 12 noon.} & \\
\end{tabular}

\bigskip
\noindent
** 5-day workshop participants are welcome to use BIRS facilities (BIRS Coffee Lounge, TCPL and Reading Room) until 3 pm on Friday, although participants are still required to checkout of the guest rooms by 12 noon. **

%%%%%%%%%%%%%%%%%%%%%%%%%%%%%%%%%%%%%%%%%%%
%
% ABSTRACTS
%
%%%%%%%%%%%%%%%%%%%%%%%%%%%%%%%%%%%%%%%%%%%

\newpage
\begin{center}

\Large{\bf 13w5083: Statistical Data Integration Challenges in Computational Biology: Regulatory Networks and Personalized Medicine}\\
\Large{\bf Aug 11 - Aug 16, 2013}\\
\end{center}

\medskip
\begin{center}
{\bf ABSTRACTS}\\
{\bf (in alphabetic order by speaker surname)}\\
\end{center}

%%%%%%%%%%%%%%%%%%%%%%%%%%%%%%%%%%%%%%%%%%%%%%%%%%%%%%%%%%%%%%%                 
%
% Copy, paste and edit the following as many times as necessary.
% You may need to use other packages and define new commands
% depending on what is sent to you by your participants.   
%
%%%%%%%%%%%%%%%%%%%%%%%%%%%%%%%%%%%%%%%%%%%%%%%%%%%%%%%%%%%%%%%

\bigskip
\noindent
Speaker: {\bf Keith Baggerly} (MD Anderson Cancer Center)\\
Title: {\it When is Reproducibility an Ethical Issue? Genomics, Personalized Medicine, and Human Error}\\
Abstract: Modern high-throughput biological assays let us ask detailed
questions about how diseases operate, and promise to let us
personalize therapy. Careful data processing is essential, because our
intuition about what the answers “should” look like is very poor when
we have to juggle thousands of things at once. When documentation of
such processing is absent, we must apply “forensic bioinformatics” to
work from the raw data and reported results to infer what the methods
must have been. We will present several case studies where simple
errors may have put patients at risk. This work has been covered in
both the scientific and lay press, and has prompted several journals
to revisit the types of information that must accompany
publications. We discuss steps we take to avoid such errors, and
lessons that can be applied to large data sets more broadly.

\bigskip
\noindent
Speaker: {\bf Alexis Battle} (Stanford University)\\
Title: {\it Characterizing the genetic basis of transcriptome diversity through RNA-sequencing}\\
Abstract: Understanding the consequences of regulatory variation in the human
genome remains a major challenge, with important implications for
understanding gene regulation and interpreting the many disease-risk
variants that fall outside of protein-coding regions.  Here, we
provide a direct window into the regulatory consequences of genetic
variation by sequencing RNA from 922 genotyped individuals. We present
a comprehensive description of the distribution of regulatory
variation – by the specific expression phenotypes altered, the
properties of affected genes, and the genomic characteristics of
regulatory variants. We detect variants influencing expression of over
ten thousand genes, and through the enhanced resolution offered by
RNA-sequencing, we identify thousands of variants associated with
specific phenotypes including splicing and allelic
expression. Evaluating the effects of both long-range intra-
chromosomal and trans (cross-chromosomal) regulation, we observe
modularity in the regulatory network, with three-dimensional
chromosomal configuration playing a particular role in regulatory
modules within each chromosome. We also observe a significant
depletion of regulatory variants affecting central and critical genes,
along with a trend of reduced effect sizes as variant frequency
increases, providing evidence that purifying selection and buffering
have limited the deleterious impact of regulatory variation on the
cell. Further, generalizing beyond observed variants, we have analyzed
the genomic properties of variants affecting both expression and
splicing, and developed a Bayesian model to predict regulatory
consequences of novel variants, applicable to the interpretation of
individual genomes and disease studies. Finally, this cohort was
interviewed extensively to record medical, behavioral, and
environmental variables, offering an opportunity to study their
effects at a large scale.  We have explored the impact of these
environmental factors on transcriptional phenotypes, in addition to
their relationship with regulatory variation, observing broad changes
correlated with time of day, substance use, and medication, including
changes in pathways relevant to disease risk. Together, these results
represent a critical step toward characterizing the complete landscape
of human regulatory variation.

\bigskip
\noindent
Speaker: {\bf Colin Begg} (Memorial Sloan-Kettering Cancer Center)\\
Title: {\it Use of Tumor Mutational Profiles to Infer Etiologic Heterogeneity of Cancers}\\
Abstract: Cancer has traditionally been studied using the disease site
of origin as the organizing framework. Recent advances in molecular
genetics have begun to challenge this taxonomy, as detailed molecular
profiling of tumors has led to discoveries of subsets of tumors that
possess distinct clinical and biological characteristics. Increasingly
investigators are examining whether sub-types defined by molecular or
other tumor characteristics have distinct etiologies. To date,
research in this field has typically involved the comparison of
individual risk factors between tumors classified on the basis of
candidate tumor characteristics or candidate sub-types. In this talk a
more general, conceptual methodologic framework is presented, with a
view to providing formal strategies for designing and analyzing
epidemiologic studies to investigate etiologic heterogeneity. A
unitary measure of etiologic heterogeneity is proposed that can be
used to define quantitatively the degree of heterogeneity exhibited by
a set of candidate tumor sub-types. It can be shown that overall risk
predictability increases monotonically with etiologic
heterogeneity. Candidate classification systems can be compared with
respect to this measure to identify sets of sub-types with high
degrees of heterogeneity. Data from case-control studies of breast
cancer will be used to illustrate the ideas and corresponding analytic
methods. It can also be shown that molecular profiles of double
primary malignancies are uniquely informative for investigating this
topic. The investigative strategy provides a structured approach to
investigating the relationship between germ-line and somatic
mutational profiles.

\bigskip
\noindent
Speaker: {\bf Scott Boyd} (Stanford University)\\
Title: {\it Monitoring Human Lymphocyte Populations with High-Throughput DNA Sequencing}\\
Abstract: Next-generation DNA sequencing (NGS) of immunoglobulin or T
cell receptor gene rearrangements provide a new method for evaluating
the diversity, clonality, and function of a patient’s B cell or T cell
populations in immune system responses to vaccination or pathogen
exposure. In addition, abnormal lymphocyte populations are a feature
of allergic and autoimmune disorders, and can be detected and tracked
using this methodology. Similarly, the immune receptor gene
rearrangements in a previously diagnosed lymphoid malignancy represent
highly specific tumor markers that can be used to monitor for
relapse. I will discuss experimental designs and data analysis
approaches that increase the interpretability and value of these
complex data sets in human clinical studies. As one example,
measurement of B cell clonal expansions in the blood following
influenza vaccination provides an early and predictive metric of
whether or not an individual will seroconvert and increase
virus-specific antibody titers. Deep sequencing data identifying
expanded B cell clonal lineages following vaccination correlate well
with the results of single cell flow cytometric sorting and
recombinant antibody synthesis identifying influenza-specific
plasmablasts, supporting the biological relevance of overall B cell
repertoire monitoring. Strikingly, B cell responses to influenza
vaccination in different people show a prominent family of
influenza-specific IgH rearrangements which are different at the DNA
sequence level but highly similar at the protein level, indicating
that convergent selection of antibodies to specific antigens is a
common feature of human immune responses.

\bigskip
\noindent
Speaker: {\bf Chad Creighton} (Baylor College of Medicine)\\
Title: {\it Pathway-level insights from The Cancer Genome Atlas (TCGA)}\\
Abstract: Sequencing and microarray-based technologies are generating
large amounts of high quality molecular data. A mandate of The Cancer
Genome Atlas (TCGA) has been to collect and make available
comprehensive genomic data sets on human cancers, representing
multiple levels of data (DNA mutation, DNA copy, DNA methylation,
mRNA, miRNA, and protein). This presentation will focus on integrative
analyses of TCGA datasets, towards the goal of providing a more
complete view of pathway deregulation in cancer.

\bigskip
\noindent
Speaker: {\bf Yoav Gilad} (University of Chicago)\\
Title :{\it Understanding gene regulation (or not)}\\
Abstract: Histone modifications are important markers of function and
chromatin state, yet the DNA elements that direct them to specific
locations in the genome are poorly understood. Here we use the genetic
variation in Yoruba lymphoblas- toid cell lines as a natural
experiment to identify genetic differences that af- fect histone marks
and to better understand their relationship with transcrip- tional
regulation. Across the genome, we identified hundreds of quantitative
trait loci that impact histone modification or RNA polymerase (PolII)
occu- pancy. In many cases the same variant is associated with
quantitative changes in multiple histone marks and PolII, as well as
in DNaseI sensitivity and nu- cleosome positioning, indicating that
these molecular phenotypes often share a single underlying genetic
cause. Polymorphisms in some transcription factor binding sites cause
differences in local histone modification and we identify specific
transcription factors whose binding leads to histone modification in
lymphoblastoid cells. Finally, we find that variants that impact
chromatin at distal regulatory sites frequently also direct changes in
chromatin and gene ex- pression at associated promoters. In summary,
the class of variants identified here generate coordinated changes in
chromatin both locally and sometimes at distant locations, frequently
drive changes in gene expression, and likely play an important role in
the genetics of complex traits.

\bigskip
\noindent
Speaker: {\bf Simon Gravel} (McGill University)\\
Title: {\it Personal genomics of the Mestizos}\\
Abstract: There is great scientific and popular interest in
understanding the genetic history of populations in the Americas.  We
wish to understand when different regions of the continent were
inhabited, where settlers came from, and how current inhabitants
relate genetically to earlier populations. Because of the important
migrations that marked the history of the continent over the last few
hundred years, many individuals derive ancestry from multiple
continental groups, predominantly African, European, and Native
American. To develop personalized medicine for such diverse
populations, we must understand how this recent admixture of
previously isolated populations impacted individual genomes.\\

\noindent I will focus on the Mestizos of Latin America and the Caribbean, and
discuss how we can integrate multiple genetic datasets to learn about
the historical processes that led to the observed diversity within
these populations, and within individuals. I will present methods to
overcome statistical challenges caused by biases in high-throughput
sequence data, propose precise estimates of the human mutation rate,
discuss the most likely origins of the Taino people, and speculate on
consequences for personalized medicine.

\bigskip
\noindent
Speaker: {\bf Kasper Hansen} (Johns Hopkins University)\\
Title: {\it A genome-wide look at DNA methylation}\\
Abstract: DNA methylation is an important epigenetic mark in mammalian
cells, implicated in tissue differentiation and cancer.  Whole-genome
bisulfite sequencing (WGBS) is a recent technological breakthrough
which has, for the first time, enabled true genome-wide measurement of
this epigenetic mark. We discuss our recent work on analyzing this
type of data and discuss changes in DNA methylation associated with
carcinogenesis as well as global differences between tissues.

\bigskip
\noindent
Speaker: {\bf David Haussler} (UC Santa Cruz)\\
Title: {\it Large-scale comparative genomics for cancer research}\\
Abstract: UCSC has built the Cancer Genomics Hub (CGHub) for the US
National Cancer Institute, designed to hold up to 5 petabytes of
research genomics data (up to 50,000 whole genomes), including data
for all major NCI projects. In its first year it has served 6
petabytes of data to more than 100 research labs.  Cancer is
exceedingly complex, with thousands of subtypes involving an immense
number of different combinations of mutations. The only way we will
understand it is to gather together DNA data from many thousands of
cancer genomes so that we have the statistical power to distinguish
between recurring combinations of mutations that drive cancer
progression and "passenger" mutations that occur by random
chance. Currently, with the exception of a few projects such as ICGC
and TCGA, most cancer genomics research is taking place in research
silos, with little opportunity for data sharing. If this trend
continues, we lose an incredible opportunity.  Soon cancer genome
sequencing will be widespread in clinical practice, making it possible
in principle to study as many as a million cancer genomes.  For these
data to also have impact on understanding cancer, we must begin soon
to move data into a global cloud storage and computing system, and
design mechanisms that allow clinical data to be used in research with
appropriate patient consent. A global alliance for sharing genomic and
clinical data is emerging to address this problem. This is an
opportunity we cannot turn away from, but involves both social and
technical challenges.\\

\noindent Reference:
http://www.eecs.berkeley.edu/Pubs/TechRpts/2012/EECS-2012-211.html

\bigskip
\noindent
Speaker: {\bf Wolfgang Huber} (EMBL)\\
Title: {\it Differential analysis of count data from high-throughput sequencing}\\
Abstract: Many applications of high throughput sequencing require
statistical inference based on count data. Mapped reads are often
summarised by counting their overlaps with genomic features of
interest (genes, exons, binding regions) in samples from different
experimental conditions. Applications include differential gene
expression, differential exon usage, HiC, ChIP-Seq, CLIP-Seq; similar
counting problems are also posed in proteomics.\\

\noindent In this talk, I will describe some of our recent work on the use of
generalised linear models of the Negative Binomial family for this
task, in particular shrinkage estimation of treatment effects and
dispersion parameters in the small sample situation, and robustness to
outlier data. Accompanying software is available in the DESeq2 package
in Bioconductor. I will also briefly present an application to the
detection of evolutionarily conserved patterns of tissue dependent
exon usage.

\bigskip
\noindent
Speaker: {\bf Laurent Jacob} (UC Berkeley)\\
Title: {\it Correcting gene expression data when neither the unwanted variation nor the factor of interest are observed}\\
Abstract: When dealing with large scale gene expression studies,
observations are commonly contaminated by sources of unwanted
variation such as platforms or batches. Not taking this unwanted
variation into account when analyzing the data can lead to spurious
associations and to missing important signals. When the analysis is
unsupervised, e.g. when the goal is to cluster the samples or to build
a corrected version of the dataset --- as opposed to the study of an
observed factor of interest --- taking unwanted variation into account
can become a difficult task. The unwanted factors may be correlated
with the unobserved factor of interest, so that correcting for the
former can remove the effect of the latter if not done carefully. We
show how negative control genes and replicate samples can be used to
estimate unwanted variation in gene expression, and discuss how this
information can be used to correct the expression data or build
estimators for unsupervised problems. The proposed methods are then
evaluated on three gene expression datasets. They generally manage to
remove unwanted variation without losing the signal of interest and
compare favorably to state of the art corrections.

\bigskip
\noindent
Speaker: {\bf Benjamin Haibe-Kains} (Institut de Recherches Cliniques de Montréal)\\
Title: {\it Prediction of Drug Response in Cell Lines: Are Pharmacogenomic Datasets Consistent?}\\
Abstract: Cancer cell line studies have long been used to test the
efficacy of therapeutic agents and to explore the genomic factors
predictive of response. Several large-scale pharmacogenomic studies
were published recently; each assayed a panel of several hundred
cancer cell lines for gene expression, copy number, genome sequence,
and pharmacological response to multiple anti-cancer drugs. The
resulting datasets present a unique opportunity to characterize
mechanisms associated with drug response. In this talk I will show
that in comparing these datasets high-throughput genomic data are well
correlated, however the measured pharmacologic response to drugs is
highly discordant. The poor correspondence is surprising as these
studies assessed drug response using common estimators: the IC50
(concentration at which the drug inhibited 50\% of the maximal
cellular growth), and the AUC (area under the activity curve measuring
dose response). For response for drugs screened in these studies, only
one drug had a correlation coefficient between studies greater than
0.6; these results are also reflected in the gene-drug associations
where inconsistent results were found. However, the results improved
when we assessed pathway-drug correspondence (a few drugs with a
correlation greater than 0.6), suggesting that analyzing the genomic
basis of drug response at the pathway level may yield greater
consistency between studies. The discrepancy in pharmacologic response
in well-controlled experiments makes drawing firm conclusions from
them very difficult and has potential implications for using these
outcome measures to assess gene-drug relationships or select potential
anti-cancer drugs.

\bigskip
\noindent
Speaker: {\bf Sunduz Keles} (University of Wisconsin)\\
Title: {\it Integrative analysis of *-seq datasets for a comprehensive understanding of regulatory roles of repetitive regions}\\
Abstract: A fundamental question in molecular biology is how cell type
specific gene expression programs are established and maintained
through gene regulation. Main drivers of cell-specific gene expression
are regulatory elements (e.g., promoters, transcription factor (TF)
binding sites, chromatin/epigenomic marks, enhancers,
silencers). Identifying genomic locations of these elements and
unraveling exactly how they control gene expression in different cell
types has been a major challenge. The ENCODE projects have generated
exceedingly large amounts of genomic data towards this end. A
formidable impediment to comprehensively understanding of these ENCODE
data is the lack of statistical and computational methods required to
identify functional elements in repetitive regions of
genomes. Although next generation sequencing (NGS) technologies, embraced by the ENCODE projects, are enabling interrogation of genomes in an unbiased manner, the data analysis efforts by the ENCODE projects have thus far focused on mappable regions with unique sequence contents. This is especially true for the analysis of ChIP-seq data in which all ENCODE-adapted methods discard reads that map to multiple locations (multi-reads). This is a highly critical barrier to the advancement of ENCODE data because significant fractions of complex genomes are composed of repetitive regions; strikingly, more than half of the human genome is repetitive.\\


\noindent We present a unified statistical model for utilizing
multi-reads in *-seq datasets (ChIP-, DNase-, and FAIRE-seq) with
either diffused or a combination of diffused and point source
enrichment patterns. Our model efficiently integrates multiple *-seq
datasets and significantly advances multi-read analysis of ENCODE and
related datasets.

\bigskip
\noindent
Speaker: {\bf Christina Kendziorski} (University of Wisconsin)\\
Title: {\it Latent Dirichlet allocation models to enable personalized genomic medicine}\\
Abstract: Genomic based studies of disease now involve highly diverse types of
data collected on large groups of patients. A major challenge facing
statistical scientists is how best to combine the data, extract
important features, and comprehensively characterize the ways in which
the features affect an individual's disease course and likelihood of
response to treatment.  In this talk, I will review methods that we
have developed to address this challenge.  Drawing an analogy from
information retrieval, we consider each patient as a document, and
data on each patient as text; and we extend the latent Dirichlet
allocation model (LDA) to our application domain. Documents are
constructed using data from multiple clinical sources and
high-throughput assays. By introducing priors that accommodate known
structure among subsets of genomic variables, the LDA based model
allows for discovery of distinct topics across the patient population
(collections of genomic aberrations, clinical variables, and
treatments) as well as determination of patient-specific mixtures over
topics.  Further model extensions provide for survival-related
responses to supervise model fit.  The approach facilitates data
integration across multiple platforms and scales to enable powerful
patient-specific inference, as demonstrated in studies of cancer from
the cancer genome atlas (TCGA) project.

\bigskip
\noindent
Speaker: {\bf Anshyul Kundaje} (MIT, Broad Institue and Stanford University)\\
Title: {\it Learning long-range regulatory interactions and unified gene regulation programs in diverse human cell-types}\\
Abstract: In multicellular organisms, epigenetic information is a key
enabler of dynamic regulatory regions shaping the identity of each
cell. This information is encoded in distinct combinations of
epigenetic modifications defining ‘chromatin states’ specific to
different types of functional elements such as promoters, enhancers,
transcribed elements and repressed domains. First, we used
multivariate Hidden Markov models to jointly learn the largest
collection of gene-proximal and distal regulatory elements from
histone modification ChIP-seq data in 120 diverse human cell-types
from the Roadmap Epigenomics and ENCODE consortia. Next, we developed
a novel probabilistic model based on Latent Dirichlet Allocation to
computationally infer putative target genes of cell-type specific
enhancers based on the associated chromatin state and gene expression
dynamics. We automatically discovered co-activated transcriptional and
enhancer modules that are strongly enriched for lineage specific
functional annotations and biochemical pathways; as well as the
complex, non-linear, cell-type specific interactions between these
modules. The resulting model showed significant improvements in
prediction of transcriptional responses compared to simple
correlation-based linking methods. The accuracy and cell-type
specificity of our predicted links were further validated by
experimental ChIA-PET chromatin interaction data in matching
cell-types and eQTL predictions. Finally, we developed a novel
ensemble learning framework based on Boosting algorithms to learn
context-specific predictive models of gene regulation by integrating
DNA binding sequence motifs of a comprehensive collection of
transcription factors with gene expression data. We dissect these
models to highlight cell-type specific regulatory elements,
transcription factors and pathways. Together, these analyses provide a
unified, multi-faceted view of dynamic gene regulation in humans.

\bigskip
\noindent
Speaker: {\bf Jeff Leek} (Johns Hopkins University)\\
Title: {\it Statistical processes for facilitating personalized medicine}\\
Abstract: The promise of personalized medicine has been tempered by
high-profile errors in the application of genomic biomarkers.Time
permitting I will discuss statistical processes for facilitating
personalized medicine: single sample normalization and artifact
correction, self-normalizing biomarkers for data integration, locked
down biomarker development, and interactive visualization. I will
illustrate these ideas with a case study in genomic biomarker
development.

\bigskip
\noindent
Speaker: {\bf Xiaole Shirley Liu} (Harvard University and Dana-Farber Cancer Institute)\\
Title: {\it Integrating sequencing and microarray data to identify novel functions of epigenetic regulators in cancer}\\
Abstract: There have been growing appreciation of the role of
epigenetic alteration in tumorigenesis and cancer progression. The
integration of recent genomic techniques and massive public data is a
useful approach to study epigenetic gene regulation in cancer. To this
end, we use chromatin dynamics from ChIP-seq and DNase-seq profiles
identify driving transcription factors in cancer progression and find
novel functions of chromatin regulators. We also integrate large scale
tumor expression data to identify novel lncRNAs with oncogenic
functions and unkown partners that mediate the novel function of
chromatin regulators.

\bigskip
\noindent
Speaker: {\bf Stephen Montgomery} (Stanford University)\\
Title: {\it The extent and impact of rare non-coding variants in humans}\\
Abstract: Recent and rapid human population expansion has led to an
excess of rare genetic variants that are expected to contribute to an
individual's genetic burden of disease risk. To date, large-scale
exome sequencing studies have highlighted the abundance of rare and
deleterious variants within protein-coding sequences.  However, in
addition to protein-coding variants, rare non-coding variants are
likely to be enriched in functional consequences. I will discuss our
effort to characterize the impact of rare non-coding variation in a
large human family and an isolated population.  Further, I will
discuss our effort to understand the systemic (multi-tissue) impact of
highly-deleterious coding variants (or variants of unknown
significance). To address this, we have developed a multiplex,
microfluidics-based method for assessing the interaction of regulatory
variation on deleterious protein-coding alleles identified through
exome sequencing. Finally, I will discuss our efforts to understand
rare and common regulatory variants underlying complex disease and
will highlight new analytical approaches for the analysis of RNA
sequencing data that we have applied to understanding cardiovascular
and lung disease.

\bigskip
\noindent
Speaker: {\bf Pierre Neuvial} (CNRS and University of Evry)\\
Title: {\it Improved performance evaluation of DNA copy number analysis methods in cancer studies}\\
Abstract: Changes in DNA copy numbers are a hallmark of cancer cells.
Therefore, the accurate detection and interpretation of such changes
are two important steps toward improved diagnosis and treatment.  The
analysis of copy number profiles measured from high-throughput
technologies such as SNP microarray and DNAseq data raises a number of
statistical and bioinformatic challenges.  Evaluating existing
analysis methods is particularly challenging in the absence of gold
standard data sets.\\

\noindent We have designed and implemented a framework to generate realistic DNA
copy number profiles of cancer samples with known parent-specific
copy-number state.  This talk illustrates some of the benefits of this
approach in a practical use case: a comparison study between methods
for segmenting SNP array data into regions of constant parent-specific
copy number.   This study helps identifying the pros and cons of the
compared methods in terms of biologically informative parameters, such
as the signal length, the number of breakpoints, the fraction of tumor
cells in the sample, or the chip type.

\bigskip
\noindent
Speaker: {\bf Roger Peng} (Johns Hopkins University)\\
Title: {\it Reproducible Research with Evidence-based Data Analysis}\\
Abstract: tatistical software is plentiful today, with new procedures and
algorithms constantly being developed, implemented, and optimized.
Traditional statistical software tends to focus on solving a
relatively self-contained task, often something that is a single piece
of a much larger data analysis. Data analysts are subsequently free to
combine the various pieces of statistical software out there in any
number of combinations to analyze their data as they see fit. Hence,
the number of "degrees of freedom" given to the analyst in most
situations is enormous. But why is this so? Statistical software is
typically written with a specific interface where certain parameters
are modifiable by the user but most others are not. A similar approach
needs to be taken at the much higher level of the entire data
analysis. Data analysis pipelines can be built using pre-determined
combinations of procedures that have been chosen based on sound
statistical evidence of their fitness or superiority.  Such analysis
pipelines---"transparent boxes"---would have relatively few options
available to the user and would be deterministic in their operation.
We call these general pipelines "deterministic statistical machines"
and present an example of one in the context of air pollution
epidemiology. We further discuss how these machines can be used to
encourage reproducible research in biomedical science.

\bigskip
\noindent
Speaker: {\bf Davide Risso} (UC Berkeley)\\
Title: {\it The role of spike-in standards in the normalization of RNA-Seq}\\
Abstract: Normalization of RNA-Seq data has proven to be an essential step to
ensure accurate inference of expression levels, by correcting for
sequencing depth and other distributional differences within and
between replicate samples. Recently, the External RNA Control
Consortium (ERCC) has developed a set of 92 synthetic spike-in
standards that are now commercially available and relatively easy to
add to a standard library preparation.
In this talk, we evaluate the performance of the ERCC spike-ins and we
use them as controls to compare different normalization strategies.
Moreover, we investigate the possibility of directly using spike-in
expression measures to normalize the data. We show that although
spike-in standards are a useful resource for evaluating accuracy in
RNA-Seq experiments, their expression measures are not stable enough
to be used to estimate even a global scaling parameter to normalize
the data.\\

\noindent We propose a novel normalization strategy that aims at removing
unwanted variation from the data by performing a factor analysis on a
suitable set of control genes and that can exploit spike-in controls
when they are present in the library, without relying exclusively on
them. Our novel approach leads to more accurate estimates of
expression fold-changes and tests for differential expression,
compared with state-of-the-art normalization methods.

\bigskip
\noindent
Speaker: {\bf Ingo Ruczinski} (Johns Hopkins University)\\
Title: {\it Sequencing family members to detect disease risk variants}\\
Abstract: We present some new statistical methods and software to
detect disease risk variants, sequencing affected only or affected and
non-affected individuals in families. Examples are mostly drawn from a
study of oral clefts with probands of Asian and European descent.

\bigskip
\noindent
Speaker: {\bf Mark Segal} (UCSF)\\
Title: {\it Reproducibility of 3D chromatin configuration reconstructions}\\
Abstract: It is widely recognized that the three dimensional (3D)
architecture of eukaryotic chromatin plays critical roles in nuclear
and cellular function.  However, until a few years ago, observing 3D
structure at even modest resolutions was problematic, because genomes
are highly condensed and assays were low-throughput.  Recently devised
high-throughput molecular techniques are changing this
situation. Notably, the development of chromatin conformation capture
(CCC) assays has enabled elicitation of “contacts”: spatially
close chromosomal loci. These techniques have provided insight into
chromatin organization at unprecedented resolutions, and permitted
exploration of the downstream influence of such organization on a
variety of biological processes, including gene regulation and
cancer-driving gene fusions. Accordingly, obtaining high resolution 3D
reconstructions of genome architecture is a compelling biological
quest. However, most analysis of CCC data has focussed on the one
dimensional contact level, with appreciably less effort directed
toward evaluating accuracy and reproducibility of 3D reconstructions,
and deploying such structures to analyze consequent biological
processes.  Questions of accuracy must be addressed experimentally.
However, questions of reproducibility can be addressed statistically.
After describing and applying a constrained optimization technique to
reconstruct chromatin configurations for a number of closely related
yeast datasets we assess the reproducibility thereof using three
relevant metrics that measure the distance between 3D configurations.
The first of these, Procrustes fitting, measures configuration
closeness after applying reflection, rotation, translation and scaling
based alignment of the structures.  The other two, congruence among
distance matrices and distance differencing, base comparisons on the
within-configuration inter-point distance matrix.  Inferential results
for these metrics rely on suitable resampling schemes.  Preliminary
findings indicate that distance matrix based approaches are preferable
to Procrustes analysis, not because of the metrics per se but rather
on account of attendant inferential (permutation) schemes.\\ 

\noindent It has recently been emphasized that the use of constrained
optimization approaches to 3D architecture reconstruction, as employed
here, can be prone to becoming trapped in local minima.  Our methods
of reproducibility assessment provide a means for comparing 3D
reconstruction solutions so that we can discern between local and
global optima by contrasting solutions under perturbed inputs.

\bigskip
\noindent
Speaker: {\bf Venkat Seshan} (Memorial Sloan-Kettering Cancer Center)\\
Title: {\it To adjust or not to adjust: the design and analysis of an epidemiologic study}\\
Abstract: Women who survive their first primary breast cancer are at an
increased risk of developing a second primary cancer in their
contralateral breast. In a recent study Reiner et al (2012) showed
that younger age at diagnosis, family history of breast cancer and
degree of relationship to affected relative are associated with the
risk of contralateral breast cancer. Of interest is whether the
molecular characteristics of the primary tumor can give a better
predictor. Since the molecular characteristics such as expression,
methylation etc. are high-dimensional feature selection is an
important step. The risk score derived from the molecular
characteristics can only be considered useful if it adds value to the
existing risk predictors. An important issue is whether the features
used in the molecular risk predictor should be selected adjusting
upfront for known risk factors or selected unadjusted and adjusted
post development of score. We will compare the performance of the two
approaches using simulations in a simple logistic regression
framework.

\bigskip
\noindent
Speaker: {\bf Ronglai Shen} (Memorial Sloan-Kettering Cancer Center)\\
Title: {\it Pattern discovery and cancer gene identification in integrated cancer genomic data}\\
Abstract: Large-scale integrated cancer genome characterization
efforts including the cancer genome atlas have created unprecedented
opportunities to study cancer biology in the context of knowing the
entire catalog of genetic alterations. A clinically important
challenge is to discover cancer subtypes and their molecular drivers
in a comprehensive genetic context.  Curtis et al. [Nature (2012)
486(7403):346­352] has recently shown that integrative clustering of
copy number and gene expression in 2,000 breast tumors reveals novel
subgroups beyond the classic expression subtypes that show distinct
clinical outcomes. To extend the scope of integrative analysis for the
inclusion of somatic mutation data by massively parallel sequencing,
we propose a framework for joint modeling of discrete and continuous
variables that arise from integrated genomic, epigenomic, and
transcriptomic profiling. The core idea is motivated by the hypothesis
that diverse molecular phenotypes can be predicted by a set of
orthogonal latent variables that represent distinct molecular drivers,
and thus can reveal tumor subgroups of biological and clinical
importance. To identify genomic features that contribute most to the
biological variation and thus have direct relevance for characterizing
the molecular subgroups, we apply a penalized likelihood approach. We
show application of the method to the TCGA pan-cancer cohort with
whole-exome DNA sequencing, SNP6.0 array, mRNA sequencing data in
3,000 patient samples spanning 12 cancer types.

\bigskip
\noindent
Speaker: {\bf Noah Simon} (Stanford University)\\
Title: {\it Estimating Many Effect-sizes Bayesian Estimation as a Frequentist}\\
Abstract: With the advent of high-throughput technologies, the multiple testing
problem has become pervasive in the analysis of biomedical and genomic
data. Much attention has been devoted to this issue, and in many cases
the field has developed good solutions. However, there is an equally
important but more overlooked problem: estimating the effect-sizes
of the significant features. The standard for estimating
effect-sizes in high-throughput problems has been the empirical
bayes methods of Robbins (Robbins [1951] and others), which was
recently brought back into the limelight by Efron (Efron [2010] and
others). Combined with new developments in flexible density estimation
these methods perform somewhat astoundingly well. Unfortunately these
methods are still not widely used in practice --- they are often seen
as unintuitive, or clever but impractical (eg. James-Stein). In this
talk I will give a simple intuitive reformulation of the frequentist
effect-size estimation problem. This reformulation will lead
directly to the empirical bayes approach. Along the way I will include
a number of simulated and real data examples. At the end I will
discuss some unresolved issues and future research directions.\\

\noindent B. Efron. Large Scale Inference: Empirical Bayes Methods for
Estimation, Testing, and Prediction. Cambridge, 2010.

\noindent H. Robbins. Asymptotically subminimax solutions of compound
statistical decision problems. Stanford-Berkeley joint symposium,
1951.

\bigskip
\noindent
Speaker: {\bf David Stephens} (McGill University)\\
Title: {\it Statistical modeling and computation for methylation profiles in the BLK gene region}\\
Abstract: I will discuss some work in progress on statistical methods for
extracting patterns from methylation profiles,
particularly sequencing data, in an example based on the BLK gene region.
The assay involves a high throughput technology that
can output strand-specific reads and methylation proportions over relatively
large regions.  Our approach
uses hidden Markov models, and I will discuss discrete and continuous latent
representations of the methylation
patterns.\\

\noindent This is joint work with Asad Haris, Celia Greenwood and Aurelie Labbe.

\bigskip
\noindent
Speaker: {\bf Barry Taylor} (UCSF)\\
Title: {\it Outlier genomics drives precision oncology }\\
Abstract: Curative therapy for patients with advanced-stage solid tumors remains
elusive. Even with the much-heralded advent of targeted inhibitors of
oncogenic signaling pathways, drug resistance and disease progression
occur in essentially all patients. Indeed, little is known about the
molecular genetic basis of exceptional and curative responses to
cancer therapy. We have begun to investigate, with whole-genome
sequencing and associated approaches, the genetic basis of complete
and durable responses to both targeted and systemic anti-cancer
therapies, an outlier phenotype. Here, we discuss early successes and
challenges as well as the opportunities for variant interpretation in
clinical specimens from patients with established phenotypes. These
studies have revealed not only individual sensitizing mutations, but
also synergistically acting genetic interactions and the contribution
of tumor clonality to the durability of treatment response. Together,
these data have yielded unprecedented insights into the molecular
genetic basis of exceptional responses, leading to the discovery of
(i) novel pathway biology, (ii) previously occult biomarkers of
clinical benefit, and (iii) rational polytherapeutic strategies to
interdict in a manner that extends such profound, life-altering
activity in molecular defined populations.
 
\bigskip
\noindent
Speaker: {\bf Pei Wang} (Fred Hutchinson Cancer Research Center)\\
Title: {\it Regularized multivariate regression approaches for integrative genomic analysis}\\
Abstract: Understanding expression quantitative trait loci (eQTL)
provides important clues to genetic basis of gene expression
regulation. In this talk, we introduce a new statistical method,
GroupRemMap, for identifying eQTLs.  We model the dependent
relationship between gene expression and single nu- cleotide variants
(SNVs) through a multivariate linear regression model, in which gene
expression levels are treated as outcomes and SNV genotypes are
treated as predictors. To handle the high-dimensionality as well as to
incorpo- rate the intrinsic group structure of SNV data, we introduce
a new regularization scheme to (1) control the overall sparsity of the
model; (2) encourage the group selection of SNVs from the same gene;
and (3) facilitate the detection of trans-hub-eQTLs. We apply the
proposed method to the colorectal and breast cancer data sets from the
cancer genome atlas (TCGA), and identify several biologically
interesting eQTLs. These findings could potentially inform the
underlying biological processes of cancers and generate hypotheses for
future studies.

\bigskip
\noindent
Speaker: {\bf Hongyu Zhao} (Yale University)\\
Title: {\it Joint analysis of expression profiles from multiple cancers to identify microRNA-gene interactions}\\
Abstract: MicroRNAs (miRNAs) play a crucial role in tumorigenesis and
development through their effects on target genes. The
characterization of miRNA-gene interactions will lead to a better
understanding of cancer mechanisms. Many computational methods have
been developed to infer miRNA targets with/without expression
data. Since expression data sets are in general limited in size, most
existing methods concatenate datasets from multiple studies to form
one aggregated dataset to increase sample size and power. However,
such simple aggregation analysis results in identifying
miRNA-geneinteractions that are mostly common across data sets,
whereas specific interactions may be missed by these methods. Recent
releases of The Cancer Genome Atlas (TCGA) data provide paired
expression profiling of miRNAs and genes in multiple tumors with
sufficiently large sample size. To study both common and cancer
specific interactions, it is desirable to develop a method that can
jointly analyze multiple cancers to study miRNA-gene interactions
without combining all the data into one single data set. In this
presentation, we describe a novel statistical method to jointly
analyze expression profiles from multiple cancers to identify
miRNA-gene interactions that are both common across cancers and
specific to certain cancers. The benefit of this joint analysis
approach is demonstrated by both simulation studies and real data
analysis of TCGA datasets. Compared to simple aggregate analysis or
single sample analysis, our method can effectively use the shared
information among different but related cancers to improve the
identification of miRNA-gene interactions. Another useful property of
our method is that it can estimate similarity among cancers through
their shared miRNA-gene interactions. This is joint work with Xiaowei
Chen and Frank Slack.

\end{document}
